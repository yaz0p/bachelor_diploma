\input{./inputs/preamble}

\begin{document}

\newcommand{\signplace}{\underline{\hspace{40mm}}}
\newcommand{\dateblank}{%
    <<\underline{\hspace{10mm}}>> \underline{\hspace{30mm}} 2024 г.%
}
\newlength{\twointerv}\setlength{\twointerv}{28.34pt}

\begin{titlepage}
    \singlespacing
    \setlength{\parindent}{0pt}
    \begin{center}
        Министерство науки и высшего образования Российской Федерации\\
        Фереральное государственное бюджетное образовательное учреждение
высшего образования\\
        Российский государственный гидрометеорологический университет\\
        (РГГМУ)\\
        Институт информационных систем и геотехнологий\\
        Направление подготовки: 09.03.03 <<Прикладная информатика>>\\
        Профиль подготовки: <<Прикладные информационные системы\
        и геотехнологии>>

    \end{center}
    \vspace{\oneinterv}
    \begin{center}
    ВЫПУСКНАЯ КВАЛИФИКАЦИОННАЯ РАБОТА\\
    (БАКАЛАВРСКАЯ РАБОТА)
    \end{center}
    \vspace{\oneinterv}
    На тему: <<Разработка приложения интеллектуального ассистента на базе
    технологий глубокого обучения.>>
    \vspace{\twointerv}

    \vfill

    \begin{tabular}{N{70mm}N{80mm}}
        Научный руководитель,\\
        к.т.н & \signplace{}Петров Я.А.\\
        \vspace{5mm}
        Исполнитель,\\студент группы ПИ-Б20-2-2 & \signplace{} Попов В.Н.
    \end{tabular}

    \vfill

    \begin{center}
        Санкт-Петербург 2024 г.
    \end{center}
\end{titlepage}
\setcounter{page}{2}
\pagestyle{}


\input{./inputs/toc}

\structel{ВВЕДЕНИЕ}
За последние десятилетия произошел огромный скачок в развитии информационных технологий:
от создания первой электронной вычислительной машины, до сложный генеративных
нейронных сетей (GAN). Сейчас компании проводят множественные исследования
для выявления возможностей таких технологий и необходимости дальнейших
инвестиций в данную отрасль. Одним из представителей GAN стали большие языковые
модели (LLM). В настоящее время лидерами в данной отрасли стали: OpenAI,
которые придумали реализовать интерфейс для взаимодействия с нейройнной сетью 
в виде чата; ПАО Сбербанк, реализовавшие отечественную LLM в условиях изоляции
и ограниченных ресурсов; Meta (признана в РФ экстремистской организацией и 
запрещена), разработавшие малую языковую модель (SLM) LLaMA, ставшая прорывом 
для энтузиастов, у которых нет таких ресурсов для реализации LLM, как у 
больших игроков рынка.

В данной выпускной квалификационной работе 
будет разработана платформа, позволяющая взаимодействовать с ресурсами предприятия,
использующая LLM/SLM для генерации релевантных ответов.

Актуальность данной темы обусловлена возможностью оптимизации многих процессов, 
увеличении производительности сотрудников, повышении лояльности клиентов.
В контексте высшего учебного заведения (ВУЗ), использование технологий подобного рода
повышает конкуретноспособность ВУЗа, что наряду с предыдущими пунктами является
положительной метрикой.

Объект исследования --- большие языковые модели (LLM).

Предмет исследования --- применение языковых моделей в бизнесе.

Цель работы --- проектирование и разработка приложения, которое позволяет
взаимодействовать с структурой предприятий посредством интерфейса.

В качестве интерфейса для пользователя была выбрана оболочка в виде чат-бота.
Чат-боты давно вошли в жизнь большинства населения. Это подтверждается 
информацией аналитической компании <<eMarketer>>, согласно которой, чат-ботами
пользуются более 1,4 млрд. человек на планете~\cite{botucount}.

Для выполнения поставленной цели были поставлены следующие задачи:
\begin{itemize}
    \item Выполнить анализ предметной области;
    \item Провести сравнительный анализ информационных систем;
    \item Рассчитать сроки реализации проекта;
    \item Смоделировать схему бизнес-процессов;
    \item Составить описание документов бизнес-процессов;
    \item Сформировать перечень требований к ИС;
    \item Исследовать подходы SWOT\@;
    \item Описать сценарии вариантов использования;
    \item Визуализировать описанные сценарии вариантов использования;
    \item Создать модель диаграммы компонентов;
    \item Создать модель диаграммы развертывания;
    \item Реализовать бизнес-логику ассистента и перенести его в интерфейс бота;
    \item Протестировать приложение;
\end{itemize}

В работе будет рассматриваться РГГМУ (далее Университет), но применяться бот
сможет не только в конкретном учебном заведении, а для любых предприятий.

Во время разработки ассистента использовалась методология Agile. Она позволила
работать в удобном темпе и формировать требования во время разработки.

В ходе выполнения практической части выпускной квалификационной
работы были использованы: Python, LangChain, FAISS, HuggingFace,
Transformers, Docker.

\sect{Предпроектный анализ}
\subsect{Анализ предметной области}

Индустрия информационных технологий является одной из наиболее динамичных и 
быстроразвивающихся отраслей, где каждый год появляются новые тенденции и
совершенствуются технологии, которые позволяют улучшить пользовательский опыт 
и приносить большую выгоду бизнесу. Одной из ключевых технологий стала 
технология трансформер, предложенная в статье “Attention is all you need”.
За недолгий промежуток времени технология успела развиться и стать
индустриальным стандартом. Увидеть применение технологии трансформер можно по
созданным продуктам: GPT-4, BERT, T5, GigaChat.

Одной из ключевых особенностей трансформеров является их способность 
обрабатывать большие объемы текста без потери информации для выполнения задач
таких как машинный перевод, обработка естественного языка, когнитивный анализ
текста и генерирование текста. Трансформер состоит из блоков кодировщиков и
декодеров, которые обрабатывают входные данные и генерируют выходные данные.
Большое количество параметров сети позволяет ей улучшить качество работы по 
сравнению с другими моделями. В последнее время наблюдается тренд
на внедрение больших языковых моделей в различные отрасли бизнеса, например
системы автоматических ответов на вопросы, чат-боты, умные помощники. 

Преимуществом таких решений является быстрый поиск информации и выдача её в
удобоваримом виде, когда без использования таких ассистентов на поиск
необходимой информации может потребоваться достаточный промежуток времени.

Не смотря на множество преемуществ данной технологии, все же присутствуют и
большие минусы: сложность обучения и масштабирования моделей, большие требования
к вычислительным ресурсам являются проблемой для массового пользователя.

В данной дипломной работе предлагается разработать информационную 
систему-помощника в виде чат-бота который будет представлять собой полезный
инструмент как для студентов, так и для сотрудников ВУЗа. Основная идея
информационной системы состоит в том, чтобы получить универсальный инструмент
для взаимодействия со всей структурой университета. В рамках чат-бота 
пользователь сможет получить всю необходимую информацию, например информацию о
заселении в общежития, списке необходимых документов для поступления и прочих
вопросах, касающихся взаимодействия с университетом.

В общем и целом, интеграция технологии больших языковых моделей является
актуальной и перспективной темой для дипломной работы, которая позволит изучить
основы построения архитектуры приложения, интеграции технологий в предприятия,
основы работы с нейросетями и машинным обучением, а также тестирования решений,
где нет очевидных метрик для измерения результата.

\subsect{Сравнительный анализ}

На данный момент прямых конкурентов у моего решения нет, но я не отрицаю того,
что в настоящий момент может разрабатываться схожее решение. Из схожих решений
можно отметить следующие решения:

Боты от университетов. Такие решения не имеют возможности масштабирования,
имеют ограниченный пул вопрос/ответ и привязанны к какой-то определенной
платформе.

Virtual Spirits. Эта зарубежная компания специализируется на создании на создании
чат-ботов для различных предприятий. Из преемуществ имеется возможность настройки
внешнего вида бота. Сравнение моего приложения и приложений конкурентов 
приведено на таблице~\ref{tbl:сравнительный анализ}.

\begin{longtbl}{сравнительный анализ}
    {Сравнительный анализ}
    {N{6cm}|N{3cm}|N{3cm}|N{3cm}}
        
    Критерий оценки & 
    \thead>{3cm}{Мой сервис} & 
    \thead>{3cm}{Боты от университетов} & 
    \thead>{3cm}{Virtual Spirits} \\\hline
\endfirsthead

\caption*{Продолжение таблицы \thetable} \\
\hline
    Критерий оценки & 
    \thead>{3cm}{Мой сервис} & 
    \thead>{3cm}{Боты от университетов} & 
    \thead>{3cm}{Virtual Spirits} \\\hline
\endhead
    \mr{3}{Масштабируемость} & \mr{3}{Высокая} & Низкая & Высокая \\\hline
    \mr{4}{Гибкость настройки} & \mr{4}{Высокая} & Низкая & Высокая \\\hline
    \mr{5}{Количество информации} & \mr{5}{Неограничено} & Ограничено & Ограничено\\\hline
    \mr{6}{Кроссплатформенность} & \mr{5}{Высокая} & Низкая & Высокая\\\hline
    \mr{5}{Стоимость обслуживания} & \mr{5}{Низкая} & Неизвестно & Высокая\\\hline
    \mr{5}{Простота интеграции} & \mr{5}{Высокая} & Средняя & Средняя\\\hline
    \mr{5}{Ручное обновление информации} & \mr{5}{Присутствует} & Отсутствует & Отсутствует\\\hline
    \mr{5}{Система авторизации} & \mr{5}{Отсутствует} & Присутствует & Присутствует
\end{longtbl}

Опираясь на проведенный анализ можно подвести некоторый итог:
в конечной системе не будет системы авторизации, так как мне кажется, что вся
информация должна быть в открытом доступе для всех возможных пользователей ИС.

Под удобным сбором информации подразумевается интутивно понятный процесс 
заполнения базы знаний, который может осуществляться как вручную, так и при 
помощи API, парсинга или других методов получения информации.

Возможность получать информацию не связанную с обучением мне кажется одним из
ключевых преимуществ моей информационной системы: для абитуриентов может быть
важно получить информацию как о возможном расписании, так и о поступлении в
военный учебный центр (ВУЦ), получении бесконтактной смарты-карты (БСК)
или же информации о истории университета.

\subsect{Системный анализ}

Исследование проектируемой ИС проводилось в виде SWOT анализа\@.

SWOT-анализ — метод стратегического планирования, суть которого заключается в 
выявлении факторов внутренней и внешней среды организации и разделении их на 
четыре категории: Strengths, Weaknesses, Opportunities, Threats.
Сильные и слабые стороны представляют факторы внутренней среды объекта анализа.
В свою очередь возможность и угрозы представляют собой внешнюю среду
объекта анализа.

SWOT анализ находится в таблице~\ref{tbl:swot}.\\\\

\begin{longtbl}{swot}
    {SWOT анализ}
    {N{2cm}|N{6.5cm}|N{6.5cm}}
        
 & \thead>{6.5cm}{Положительное влияние} 
 & \thead>{6.5cm}{Негативное влияние}  \\\hline

\endhead
\mr{1}{Внутренняя среда} & \mr{1}{--- Актуальность\\--- Простота использования}
& --- Нейросетевые галюцинации  \\
&& --- Необходимость тщательно прорабатывать интеграцию во избежание
проблем с безопасностью \\\hline
\mr{2}{Внешняя среда} & \mr{2}{--- Упрощение навигации по ресурсам ВУЗа\\
--- Получение поддержки от государства }
& --- Регуляции со стороны государства  \\
&& --- Бюрократия с какой-либо стороны
\end{longtbl}

Из положительных аспектов можно выделить простоту конечного использования и
улучшение взаимодействие с предприятием. 

Из отрицательных факторов стоит выделить следующие аспекты: нейросетевые 
галюцинации, проблемы с безопасность и бюрократия. 

Нейросетевые галюцинации --- аномалия, возникающая во время работы нейронной 
сети, влияющая на вывод результат непредсказуемым образом. Например, спросив
о технической оснащенности предприятия нейросеть может начать галлюцинировать и
ответить, что у предприятия есть несколько квантовых суперкомпьютеров, хотя это 
не так. Частично решить эту проблему может грамотный промт-инженеринг, а так
же правильное подмешивание контекста в сам промт.

Проблемы с безопасность можно отнести в ту же категорию. Если занести секретные
данные в базу знаний, то велик шанс утечки информации и попадание её в руки 
злоумышленников. Для обхода этой проблемы нужно внимательно относиться к
информации, которую вы собираетесь хранить в базе знаний и иметь базовые навыки
кибербезопасности.

Бюрократия же не позволит внедрить информационную систему, занести все 
необходимые данные без множества согласований и утверждений со стороны 
вышестоящего руководства.

\subsect{Требования к сервису}\\

Функциональные и нефункциональные требования должны быть определены до начала
реализации ИС, чтобы получить представление о конечном продукте.

Для классификации требований использовалась модель FURPS\@. В следующих 
таблицах были приведены функциональные и нефункциональные требования ИС. Эти 
требования характеризуют поведение ИС. Функциональные требования приведены 
в таблице~\ref{tbl:freq}.

\begin{longtbl}{freq}
    {Функциональные требования}
    {N{2.5cm}|N{14cm}}
        
Номер требования & \thead>{14cm}{Описание} \\\hline
\endfirsthead

\caption*{Продолжение таблицы \thetable} \\
\hline
Номер требования & \thead>{14cm}{Описание} \\\hline
\endhead
\mr{1}{FUN\_1} & При начале общения бот должен представиться и рассказать о
своих возможностях \\\hline

\mr{1}{FUN\_2} & По требованию пользователя бот должен предоставить контактную
информацию вышестоящего руководства\\\hline

\mr{1}{FUN\_3} & По требованию пользователя бот должен предоставить информацию
о расписании\\\hline

\mr{1}{FUN\_4} & По требованию пользователя бот должен предоставить информацию
о списке направлений, на которые можно поступить с определенными предметами\\\hline

\mr{1}{FUN\_5} & По требованию пользователя бот должен предоставить информацию
о университете\\\hline

\mr{1}{FUN\_6} & По требованию пользователя бот должен предоставить актуальную 
информацию о местах проведения практики\\\hline

\mr{1}{FUN\_7} & По требованию пользователя бот должен предоставить информацию 
о проводимых мероприятиях внутри университета\\

\end{longtbl}

В таблице~\ref{tbl:soft} приведены нефункциональные требования, затрагивающие
удобство использования приложения.

\begin{longtbl}{soft}
    {Удобство использования}
    {N{2.5cm}|N{14cm}}
        
Номер требования & \thead>{14cm}{Описание} \\\hline
\endfirsthead

\caption*{Продолжение таблицы \thetable} \\
\hline
Номер требования & \thead>{14cm}{Описание} \\\hline
\endhead

\mr{1}{NFR\_1} & Пользователю предоставляется информация о сервисе \\\hline

\mr{1}{NFR\_2} & Использование сервиса не требует от пользователя какого-либо 
обучения\\\hline

\mr{1}{NFR\_3} & Бот предоставляет как текстовый, так и голосовой интерфейс
для общения\\\hline

\mr{1}{NFR\_4} & По требованию пользователя бот должен предоставить информацию
о списке направлений, на которые можно поступить с определенными предметами\\\hline

\mr{1}{NFR\_5} & Информация, предоставляемая ботом должна быть избыточной\\

\end{longtbl}


В таблице~\ref{tbl:reliability} приведены нефункциональные требования, затрагивающие
надежность сервиса.

\begin{longtbl}{reliability}
    {Надежность сервиса}
    {N{2.5cm}|N{14cm}}
        
Номер требования & \thead>{14cm}{Описание} \\\hline
\endfirsthead

\caption*{Продолжение таблицы \thetable} \\
\hline
Номер требования & \thead>{14cm}{Описание} \\\hline
\endhead

\mr{1}{REL\_1} & Сервиса должен работать 24 часа в сутки 7 дней в неделю, за 
исключением технических перерывов. \\\hline

\mr{1}{REL\_2} & Точность предоставляемой информации зависит от предоставляемых
организацией данных\\

\end{longtbl}

В таблице~\ref{tbl:capacity} приведены нефункциональные требования, затрагивающие
производительность сервиса.

\begin{longtbl}{capacity}
    {Производительность сервиса}
    {N{2.5cm}|N{14cm}}
        
Номер требования & \thead>{14cm}{Описание} \\\hline
\endfirsthead

\caption*{Продолжение таблицы \thetable} \\
\hline
Номер требования & \thead>{14cm}{Описание} \\\hline
\endhead

\mr{1}{PER\_1} & На генерацию ответа у сервиса должно уходить не более
10 секунд\\\hline

\mr{1}{PER\_2} & Для работы боту необходимо: постоянный выход в сеть интернет,
    4ГиБ оперативной памяти. Так же, опционально, чтобы бот был полностью автономен
    необходима видеокарта уровня RTX 3060 и выше для размещения сервиса на SLM
    В противном случае необходимо пользоваться посредником в виде LLM от Сбербанк,
    OpenAI, Mistral и т.п.
\end{longtbl}

В таблице~\ref{tbl:support} приведены нефункциональные требования по поддержке
сервиса, описывающие удобство применение конечного приложения для заказчика.

\begin{longtbl}{support}
    {Поддержка сервиса}
    {N{2.5cm}|N{14cm}}
        
Номер требования & \thead>{14cm}{Описание} \\\hline
\endfirsthead

\caption*{Продолжение таблицы \thetable} \\
\hline
Номер требования & \thead>{14cm}{Описание} \\\hline
\endhead

\mr{1}{SUP\_1} & Установка сервиса осуществляется при помощи сценариев\\\hline

\mr{1}{SUP\_2} & Сервис должен поддерживать как работу с различными типами 
данных, поступающими от предприятия
\end{longtbl}

В таблице~\ref{tbl:safety} приведены нефункциональные требования к безопасности
сервиса.

\begin{longtbl}{safety}
    {Требования к безопасности}
    {N{2.5cm}|N{14cm}}
Номер требования & \thead>{14cm}{Описание} \\\hline
\endhead
\mr{1}{SAF\_1} & Сервис должен использовать протокол HTTPS для обмена
информацией\\
\end{longtbl}

Изложенные требования в полной мере описывают все аспекты приложения и позвляют
реализовать корректную работу.

\subsect{Регламентирующие документы}

Для соблюдения законодательства, соответствия требования заказчика,
соответствия стандартам качества и урегулирования многих аспектов работы 
программного обеспечения используются регламентирующие документы: ГОСТ,
пользовательское соглашение, рабочие инструкции и прочие.

Список регламентирующих документов приведен в таблице~\ref{tbl:reg_doc}.

\begin{longtbl}{reg_doc}
    {Регламентирующие документы}
    {N{1cm}|N{9cm}|N{2cm}|N{2cm}}
        
    № & \thead>{9cm}{Наименование документа}&
    \thead>{2cm}{Внутренний документ}&
    \thead>{2cm}{Внешний документ} \\\hline
\endfirsthead
\caption*{Продолжение таблицы \thetable} \\
\hline
    № & \thead>{9cm}{Наименование документа}&
    \thead>{2cm}{Внутренний документ}&
    \thead>{2cm}{Внешний документ} \\\hline
\endhead
    \mr{1}{1} & Политика приватности & + & -\\\hline
    \mr{1}{2} & ГОСТ 17657-79 Передача данных. & - & +\\\hline
    \mr{1}{3} & ГОСТ 34.321-96. Информационные технологии. Система стандартов по базам данных. Эталонная модель управления данными & - & +\\\hline
    \mr{1}{4} & ГОСТ Р 51904-2002. Программное обеспечение встроенных систем & - & +\\\hline
    \mr{1}{5} & ГОСТ 19.201-78. Техническое задание. требования к содержанию и оформлению & - & +\\\hline
    \mr{1}{6} & ГОСТ 19.401-78. Текст программы. требования к содержанию и оформлению & - & +\\\hline
    \mr{1}{7} & ГОСТ 34.201-89. Виды, комплектность и обозначение документов при создании автоматизированных систем & - & +\\\hline
    \mr{1}{8} & ГОСТ Р 50779.30-95. Приемочный контроль качества & - & +\\

\end{longtbl}

\subsect{Сроки реализации проекта}\\
Для оценки требуемых ресурсов – времени и денег, разработаем график работ. 
Распишем основные блоки работ – это анализ, проектирование, разработка и
документирование. Таким образом получаем, что на разработку системы потребуется
259 дней, что является довольно быстрым сроком. Для оценки стоимости проекта
возьмем средние зарплаты специалистов и получим что на оплату труда уйдет 1,38
миллиона рублей. График работ представлен на рисунке~\ref{Диаграмма временных затрат.png}
позволяет наглядно оценить длительность и стоимость работ.

\fig[1.0]{Диаграмма временных затрат.png}{Диаграмма затрат}

Так же для наглядности временных затрат была использована <<диаграмма Ганта>>,
позволяющая наглядно отследить каждый этап разработки информационной системы.
Диаграмма Ганта представленна на рисунке~\ref{Диаграмма Ганта.png}.

\sect{Проектирование информационной системы}
\subsect{Концептуальное проектирование}

Для представления о работе системы была разработа концептуальная модель
итогового приложения, представленная на рисунке
~\ref{Концептуальная модель.drawio.png}.

\fig[0.187]{Концептуальная модель.drawio.png}{Концептуальная модель}

Для отображения функциональности системы используется диаграмма вариантов
использования, представленная на рисунке~\ref{Диаграмма вариантов использования.drawio.png}.
\fig{Диаграмма вариантов использования.drawio.png}{Диаграмма вариантов использования}

Актор \emph{Пользователь} является обобщением клиентов, которые использует 
систему. Прецедент Приветствует отражает требование FUN\_1. Этот и другие
прецеденты описывают субъект Бот, который на диаграмме выделен рамкой.
Прецедент Рассказывает о возможностях отражает требование FUN\_1.
Актор Образовательная организация является источником данных, которые 
необходимы субъекту. Прецедент \emph{Информация о организации} отражает
функциональные требования: FUN\_2, FUN\_3, FUN\_4, FUN\_5, FUN\_6, FUN\_7, и
используется для логического объединения других прецедентов, для уменьшения 
количества связей на диаграмме и облегчить ее восприятие.

Для понимания перемещения данных внутри приложения была разрабработана схема
перемещения данных, представленная на рисунке
~\ref{Схема перемещения данных.drawio.png}.
\fig[0.3]{Схема перемещения данных.drawio.png}{Схема перемещения данных}

\subsect{Диаграмма компонентов}

Диаграмма компонентов описывает физическое представление системы и является
структурной диаграммой языка унифицированного моделирования. Она определяет
архитектуру разрабатываемой системы, устанавливая зависимости между 
программными компонентами. Кроме того, она предоставляет разработчикам и
архитекторам общую картину архитектуры системы, помогает им лучше понимать ее
структуру и взаимосвязи, а также является полезным инструментом для
коммуникации и документирования архитектурных решений. Разработанная диаграмма
представлена на рисунке~\ref{Диаграмма компонентов.png}.

\fig{Диаграмма компонентов.png}{Диаграмма компонентов}

\subsect{Диаграмма развертывания}

Диаграмма развертывания – это тип UML-диаграммы, которая показывает архитектуру
исполнения системы, включая такие узлы, как аппаратные или программные среды
исполнения, а также промежуточное программное обеспечение, соединяющее их. 
Для каких задач строят диаграммы развертывания:

\begin{enumerate}
    \item Визуализация полной структуры исходного кода
    \item Визуализация узлов системы для определения слабых мест
    \item Обеспечение многократного использования отдельных фрагментов
        программного кода
\end{enumerate}

Диаграмма разрабатываемой информационной системы содержит несколько узлов: 
сервер базы знаний внутри университета, сервер приложения, сервер мессенджера,
на котором базируется бот, конечный аппарат, при помощи которого будет 
производиться взаимодействие с информационной системой. Наблюдать разработанную
диаграмму развертывания можно на рисунке~\ref{Диаграмма развертыванияя.png}.

\fig{Диаграмма развертыванияя.png}{Диаграмма развертывания}


\subsect{Диаграмма последовательности}

Пайплайн действий таков: пользователь вводит запрос в текстовом формате или 
формате аудио, если сообщение передано в аудиоформате, то происходит 
расшифровка аудиозаписи, после чего полученный текст преобразуется в векторное
представление, далее происходит извлечение подходящих документов методом
ближайших соседей (KNN) из базы знаний, сформированной из документов в базе 
данных. Листинг поведения для различных входящих данные приведен в приложении Д.

Далее, наиболее релевантные документы подмешиваются к запросу языковой модели,
после чего языковая модель генерирует ответ, опираясь на полученный контекст.

Последовательность работы приложения приведена в приложении
~\ref{Диаграмма последовательности.png}.

\subsect{Схема базы данных}

В качестве базы данных для хранения изначальных документов для построения 
retirivial использовалась система управления базами данных (СУБД) PostgreSQL\@.
Данное решение обладает рядом преимуществ: высокая производительность при 
работе с большими объемами информации; PostgreSQL является свободным 
программным обеспечением, что позволяет использовать данное ПО для любых 
нужд на бесплатной основе; удобство использования и возможность удобного 
переноса данных из других баз данных. Данные пункты стали ключевыми при выборе 
СУБД для хранения информации. 

Такой выбор СУБД позволяет обеспечить высокую отказоустойчивость. Схема 
хранения данных внутри базы данных указана на рисунке
~\ref{Схема базы данных.png}

\fig[0.3]{Схема базы данных.png}{Схема хранения документов}

\subsect{Векторная база данных}

Для хранения векторов нужна особая база данных: необходимо быстро находить
релевантные вектора и нужно обрабатывать данные любой размерности.
Для решение этой задачи было выбрано решение от Meta (признана в РФ
экстремистской организацией и запрещена) --- FAISS. FAISS является легковесным
решением, что идеально кореллирует с моим проектом, в котором не будет огромного
количества векторов; не требует большого количества ресурсов; внутри используется
хороший поисковый движок; можно использовать ресурсы как GPU, так и CPU, что
является плюсом при отсутствии мощного GPU, или для разделения нагрузки при
использовании SLM\@. Схема векторизации документов приведена на рисунке
~\ref{Схема векторизации документов.png}.

\fig{Схема векторизации документов.png}{Схема векторизации документов}

\subsect{Развертывание приложения}
Планируется, что приложение будет размещаться на мощностях образовательного
учреждения. Для упрощения внедрения было принято решение разработать скрипт
для контейнеризации приложения.

В качестве базового образа был выбран образ с GNU Linux в качестве
операционной системы.
GNU Linux является свободным программым обеспечением, что позволяет
использовать его для любых нужд. На листинге 2.1 отображена конфигурация
Dockerfile.

\begin{codepiece}{in-place code}{Dockerfile}
    FROM python:3.12.3-bullseye
    
    WORKDIR /app
    
    COPY . .
    
    RUN pip install -r --no-cashe-dir req.txt
    
    RUN apt-get update && apt-get install -y nano
    RUN apt-get update && apt-get install -y wget bzip2 libxtst6 libgtk-3-0\
     libx11-xcb-dev libdbus-glib-1-2 libxt6 libpci-dev
    
    CMD ["/bin/bash", "-c", "python main.py"]
\end{codepiece}

Данный файл позволяет создать контейнер, содержащий все необходимые 
необходимые для работы зависимости. Использование изолированного окружения,
выделенных ресурсов позволяют сервису автономно функционировать на протяжении
долгого времени без необходимости обслуживания, что соответствует
нефункциональному требованию REL\_1. Так же созданная конфигурация для создания
изолированного контейнера позволяет запустить приложение и все необходимые
зависимости при помощи одной команды, что отражает нефункциональное требование
по отношению к поддержке сервиса SUP\_1.

\sect{Разработка системы}
\subsect{Выбор средств разработки}

В качестве языка программирования для разработки системы был выбран Python. 
Выбор связан с набором ПО, необходимым для функционирования сервиса, который
был определен на последнем этапе проектирования. Также Python является
наиболее подходящим языком для работы с искусственным интеллектом, для
него имеется много библиотек, нацеленных на решение конкретных проблем.

В качестве системы управления версиями был выбран Git. В качестве системы 
управления репозиториями был выбран GitHub. С недавнего времени GitHub 
располагает возможностью для создания бесплатных приватных репозиториев ---
никто, кроме авторизовазанных пользователей не будет иметь доступа к вашему
репозиторию.

В качестве системы для написания кода был выбран текстовый редактор NeoVim.
NVim ориентирован на разработку без мыши, что позитивно влияет на скорость 
разработки; можно запустить без графического окружения рабочего стола, что
позволяет использовать данный текстовый редактор как на домашнем персональном
компьютере, так и на сервере, не имеющим графической оболочки. Конфигурация
приведена на листинге Б.1.

Для верстки PDF-документа был использован \LaTeX. С использованием пакетов
данный инструмент позволяет достичь отличных результатов для написания
текстовых документов. Так же данное программное обеспечение является свободным.
Частичная конфигурация \LaTeX будет приведена в приложении Г.
\subsect{Структура приложения}

Созданное приложение имеет модульную структуру. Это позволяет распределить
зоны ответственности каждого модуля~\cite{solid}; к тому же это позволяет повысить 
отказоустойчивость конечного приложения. 

Схема файловой структуры показана на листинге 3.1.

\begin{codepiece}{in-place code}{Файловая структура}
|-- Dockerfile
|-- documents
|------about.txt
|------checking.txt
|--  common_questions.txt
|--  deductions_recoveries.txt
|-- embeding_store
|------index.faiss
|------index.pkl
|-- LICENSE
|-- main.py
|-- messages.log
|-- README.md
|-- req.txt
|-- retriveval.py
|-- speech2text.py
|-- tex
|-- text_generator.py
|-- videohandler.py
\end{codepiece}

Файл \emph{Dockerfile} содержит инструкцию для сборки контейнера и последующего
запуска приложения внутри контейнера.

Директория \emph{documents} содержит основные документы в текстовом формате,
предназначенные для дальнейшего перевода их в векторное представление.
Документы представляют собой файлы в формате .txt, хранящие в себе информацию
о предприятии, например, информацию о военном учебном центре, которые в
дальнейшем будут использоваться для создания векторного пространства.
По мере развития проекта данный список может меняться.

Директория \emph{embeding\_store} содержит в себе: созданное векторное 
пространство, метаданные, необходимые для поиска, индексы векторов, структуры
данных, параметры.

Файл \emph{main.py} запускает основное приложение: запуск retrivial, speech2text,
запуск логирования, а так же телеграмм клиента.

Файл \emph{retrivial.py} несет в себе следующий смысл: перевод корпуса текстов
в векторное представление, извлечение 3х наиболее релевантных документов для
составления контекста, а также обогащение запроса пользователя извлеченной
информацией, для составления ответа.

Файл \emph{speech2text.py} представляет собой блок расшифровки в аудио
в текстовый формат, для использования блоком \emph{retrivial.py}, если
это потребуется.

Файл \emph{videohandler.py} представляет собой блок разделения видео- и
звуковых дорожек. Извлеченная звуковая дорожка используется блоком
\emph{speech2text.py}.


\subsect{Расчет надежности программного и аппаратного обеспечения}

Для расчета надежности реализуемого продукта необходимо выявить элементы
системы, которые могут выйти из строя. Для себя я поделил их на 3 категории, и
составил таблицы~\ref{tbl:hardware},~\ref{tbl:software} и~\ref{tbl:stochastic}.
Программная часть, аппаратная часть, независящие от обстоятельств. Данные для
столбца 'Коэффициент надежности' брались из открытых источников.

\begin{longtbl}{hardware}
    {Аппаратная часть}
    {N{2.5cm}|N{3cm}|N{4cm}|N{5cm}}
        
    Номер & \thead>{3cm}{Коэффициент надежности}&
    \thead>{4cm}{Краткое описание}&
    \thead>{5cm}{Что повлечет?} \\\hline
\endfirsthead
\caption*{Продолжение таблицы \thetable} \\
\hline

    Номер & \thead>{3cm}{Коэффициент надежности}&
    \thead>{4cm}{Краткое описание}&
    \thead>{5cm}{Что повлечет?} \\\hline
\endhead
    \mr{1}{1} & 0.99 & Выход из строя жесткого диска (HDD)& Поломка HDD влечет за собой:
потерю данных, простой системы.\\\hline

\mr{1}{2} & 0.99 & Выход из строя процессора (CPU) & Поломка CPU влечет за собой: 
простой системы; если мы используем сервер, то потерю производительности; 
большие траты на замену и обслуживание\\\hline

\mr{1}{3} & 0.87 & Выход из строя графического процессора (GPU) & Выход из строя GPU грозит:
снижением производительности или полным выходом из строя системы, большие
затраты на замену и обслуживание\\\hline

\mr{1}{4} & 0.85 & Выход из строя оперативной памяти & Выход из строя модуля оперативной памяти
не так критичен для системы, модулей памяти несколько и вышел один из них, но
если сломались все плашки оперативной памяти, то система не сможет функционировать.\\\hline

\mr{1}{5} & 0.95 & Выход из блока питания & Выход из строя блока питания сулит
полный выход строя системы
\end{longtbl}


\begin{longtbl}{software}
    {Программная часть}
    {N{2.5cm}|N{3cm}|N{4cm}|N{5cm}}
        
    Номер & \thead>{3cm}{Коэффициент надежности}&
    \thead>{4cm}{Краткое описание}&
    \thead>{5cm}{Что повлечет?} \\\hline
\endfirsthead


\caption*{Продолжение таблицы \thetable} \\
\hline
    Номер & \thead>{3cm}{Коэффициент надежности}&
    \thead>{4cm}{Краткое описание}&
    \thead>{5cm}{Что повлечет?} \\\hline
\endhead

    \mr{1}{1} & 0.93 & Выход из строя модуля, ответственного за перевод звука в текст& 
Выход из строя данного модуля не полечет за собой больших проблем для обычных
пользователей, так как будет доступен обычный текстовый ввод, но для людей с
ограниченными возможностями перестанет быть возможным использование моей 
системы\\\hline

\mr{1}{2} & 0.84 & Выход из строя модуля, ответственного за перевод документов в 
векторное представление & Поломка данного модуля может делиться на две категории:
перестает работать перевод текста в векторное представление и перестает работать
извлечение векторов. В первом случае не все так критично, так как часто обновлять
список актуальных документов не нужно, ведь большая часть информации не меняется.
Если же сломается модуль, ответственный за извлечение векторных представлений,
то система в большей своей части перестанет работать, как задумывалось, но все 
равно будет работать: языковая модель будет выдавать ответы, но не будет иметь
контекста о университете, что негативно повлияет на точность ответа.\\\hline

\mr{1}{3} & 0.77 & Выход из строя модуля клиентского приложения & Выход из строя
клиентского приложения сулит собой: репутационные риски, недоступность приложения
для пользователей

\end{longtbl}


\begin{longtbl}{stochastic}
    {Стохастичные переменные}
    {N{2.5cm}|N{3cm}|N{4cm}|N{5cm}}
        
    Номер & \thead>{3cm}{Коэффициент надежности}&
    \thead>{4cm}{Краткое описание}&
    \thead>{5cm}{Что повлечет?} \\\hline
\endfirsthead
%\caption*{Продолжение таблицы \thetable} \\
%\hline
    \mr{1}{1} & 0.99 -- 0.97 & Массовое отключение интернета & Без подключения к сети
интернет система полностью перестает работать. Это можно будет решить, если 
поднять локальную сеть\\\hline

    \mr{1}{2} & 0.99 -- 0.97 & Отключение электричество & Без электричества система полностью 
перестает работать\\

\end{longtbl}

Расчитывая стохастические переменные я отталкивался от того, что выключить
интернет или электричество на сутки могут от 2 до 10 раз

Так как моя система избыточна: множество модулей позволяют работать другим
модулям при выходе из строя, то я буду считать отказоустойчивость системы по
следующей формуле: $R_{total} = 1 - (1 - R_1)(1 - R_2)\ldots(1 - R_n)$, где 
$R_{total}$ --- общая отказоустойчивости системы, $R_n$ --- отказоустойчивость
компонента системы. 

По расчетам надежность аппаратно-программного комплекса
$\approx 1 - 2.5116e-14$, что я считаю довольно хорошим показателем для
такой системы.

\subsect{Расчет ожидаемого результата экономической эффективности}

Основными затратами для реализации информационной системы являются:

\begin{itemize}
\item покупка необходимого аппаратного обеспечения для обеспечения работы
    языковой модели
\item в случае, если предприятие не хочет или не может позволить аппаратное 
    обеспечение для развертывания системы на своих мощностях, то оформление 
    подписки у предприятий и аренда мощностей для получения доступа к языковой 
    модели
\item обслуживание сервера
\end{itemize}

Предполагается несколько концепций для получения прибыли от конечного продукта:
Продажа подписки коммерческим и некомерческим организациям, по предоставлению 
конечного продукта. Для некоммерческих организаций подписка будет стоить от 
5000 рублей в месяц, до 15000 рублей в месяц. В свою очередь для коммерческих
организаций планируется разрабатывать уникальный договор по оформлению подписки
на конечный сервис, для получения максимальной выгоды от сделки.

К подписочной модели получения прибыли так же не исключается возможность продажи
программного обеспечения. В таком случае цена за конечный продукт так же 
определяется уникальным договором. В таком случае конечный потребитель не 
платит за подписку и сам обслуживает всю систему, но имеет над ней полный 
контроль. Со стороны продавца передается документация к программному обеспечению.

Для расчета возврата инвестиций используется формула~\ref{roi}:

\begin{equation} \label{roi}
    ROI = \frac{PROFIT - COST}{COST} \cdot 100\%
\end{equation}

Если вести подсчет для реального случая, то в худшем случае получается ситуация,
приведенная в формуле~\ref{realroi}:
\begin{equation} \label{realroi}
    ROI = \frac{60000 - 15000}{15000} \cdot 100\% \approx 300\%
\end{equation}
что равно $300\%$. Но в эти цифры не заложена покупка аппаратного комплекса, 
так как рассматривается ситуация, что я у предприятия есть мощности для 
размещения малой языковой модели.

\subsect{Word2vec}

Для хранения представлений документов в понятном для машины виде используется
подход word2vec.
Подход word2vec имеет две основные реализации: Continuous Bag of Words (CBOW)
и skip-gram.

Подход CBOW работает следующим образом: на вход системе подается несколько слов
из контекста, где количество слов задается параметром window size; скрытый слой
принимает полученные векторные представления слов, складывает их и усредняет;
на выходе получается вероятностное распределение для целевого слова.
Обучение данной модели заключается в минимизации функции потерь между
предсказанным и истинным словом. Для этого применяется градиентный спуск.
Визуализацию работы градиентного спуска можно наблюдать на рисунке
~\ref{Градиентный спуск CBOW.png}.

\fig{Градиентный спуск CBOW.png}{Визуализация работы градиентного спуска }

Skip-gram работает иначе, чем CBOW\@\cite{skip}: на вход подается единственное слово, после
чего скрытый слой преобразует данное слово в вектор; на выходе получаем
матрицу, содержащую вероятность появления какого-либо слова рядом с ним. 
Как и в случае с CBOW обучение представляет собой алгоритм обратного 
распространения ошибки.

Преобразование текстовых корпусов в вектора работает следующим образом:
после передачи на вход программе текста программа создает для каждого слова
вектор, после чего происходит процесс обучения. Во время обучения происходит
попытка предсказать контекст слова, основываясь на его окружении. Например,
если в тексте явно кореллируют слова ``упряжка`` и ``конь``, то эти векторные
представления данных слов будут иметь сходство. Визуализацию данного свойства
можно наблюдать на рисунке~\ref{Word_vector_illustration.png}

\fig[0.6]{Word_vector_illustration.png}{Взаимосвязь векторов слов}

В конце обучения создается векторное пространство, используемое в дальнейших
этапах работы приложения.

Так как большинство решений не оптимизированы для применения на домашних
персональных компьютерах (ПК) было принято решение переопределить некоторые методы
векторизатора из библиотеки HuggingFace, чтобы улучшить работу на своем ПК.
На листинге 3.2 отображены переопределенные методы. Созданое решение отражает
нефункциональное требование, относящееся к поддержке сервиса --- SUP\_2.

\begin{codepiece}{in-place code}{Векторизатор}
    class HuggingFaceE5Embeddings(HuggingFaceEmbeddings):
        def embed_query(self, text: str) -> List[float]:
            text = f"query: {text}"
            return super().embed_query(text)
    
        def embed_documents(self, texts: List[str]) -> List[List[float]]:
            texts = [f"passage: {text}" for text in texts]
            return super().embed_documents(texts)
    
        async def aembed_query(self, text: str) -> Coroutine[Any, Any, List[float]]:
            text = f"query: {text}"
            return await super().aembed_query(text)
    
        async def aembed_documents(
            self, texts: List[str]
        ) -> Coroutine[Any, Any, List[List[float]]]:
            texts = [f"passage: {text}" for text in texts]
            return await super().aembed_documents(texts)
\end{codepiece}

\subsect{Поиск подходящих векторов}

Созданное векторное пространство делится на множество кластеров
методом K-средних. Для каждого кластера выставляется центроида для
определения сущности каждого корпуса векторов. Это позволяет не перебирать все
вектора, а проходиться по принадлежности к какой-то сущности. При большом
наборе данных это позволяет быстрее находить необходимый вектор.
На рисунке~\ref{Кластеризация векторов.jpg} схематично отображен алгоритм
кластеризации.

\fig{Кластеризация векторов.jpg}{Визуализация кластеризации векторов}

Наиболее релевантные вектора~\cite{vec} сопоставляются при помощи метода K-ближайших 
соседей. Евклидово расстояние между векторами вычисляется по приведенной формуле
~\ref{eqrange}:\\
\begin{equation} \label{eqrange}
    d(x,y)=\sqrt{\sum_{i=1}^{n}{(x_i-y_i)}^2}
\end{equation}

Так же для поиска похожих векторов используется косинусное расстояние
~\ref{cosrange}:

\begin{equation} \label{cosrange}
    \cos(\theta) = \frac{A \cdot B}{\| A\| \|B\|} = 
    \frac{\sum_{i=1}^{n}A_i \times B_i}{\sqrt{\sum_{i=1}^{n}(A_i)^2}
    \times \sqrt{\sum_{i=1}^{n}(B_i)^2}}
\end{equation}
И хотя данный подход не покрывает все возможные случаи~\cite{leven}, он
является наиболее универсальным.

Визуализацию работы поиска релевантных векторов можно на рисунке
~\ref{Визуализация KNN.png}
\fig[0.769]{Визуализация KNN.png}{Визуализация нахождения релевантных векторов}

\subsect{Генерация текста при помощи поиска и аугментаций}

Для работы системы необходимо подать на вход какой либо текст.
В качестве клиента было выбрано клиентское приложение Telegram, а в качестве
интерфейса для взаимодействия приложения и пользователя TelegramBotAPI, но
в качестве клиента можно использовать и другие платформы: ВК, Яндекс.Алиса,
Одноклассники и прочие. За последнее время боты стали очень популярным
решением,~\cite{bots} благодаря удобности и доступности, что позволяет
бизнесу удобно использовать подобные интерфейсы в своих интересах.
Чтобы пользователю было удобно использовать созданное приложение было
принято описать идею приложения, а так же дать ссылку на вызов
функции-помощника. Данный аспект отражает нефункциональные требования
NFR\_1 и NFR\_2.

На листинге 3.3 отображена реализация приветственного сообщения от бота.

\begin{codepiece}{in-place code}{Стартовое сообщение}
    dummy_message = "Ищу ответ..."
    stub = hbold("Бот работает в тестовом режиме – не все ответы могут быть достоверными")
    # Start message
    @dp.message(CommandStart())
    async def command_start_handler(message: Message) -> None:
        """
        This is handler for command `/start`
        """
        await message.answer(
            f"Привет, {hbold('студент')}\n\nПеред тобой бот-асситент ИИСИГТ,\
     подготовленный {hlink('мной', 'https://github.com/yaz0p')} к качестве\
     выпускной квалификационной работы\n\nДанный бот позволяет осуществлять\
     удобную навигацию внутри университета и отвечает на не самые очевидные\
     вопросы. \n\nБот работает на основе большой языковой модели, а\
     информацию о внутренних источниках получает благодаря {hbold('RAG')}.\
     Возможно, я прикреплю сюда git репозиторий с реализацией проекта,\
     чтобы студенты могли изучить работу данного ассистента и/или дополнить\
     его своими идеями.\n\n\
    Пример того, что ты можешь спросить:\n- Адрес какого либо-корпуса\n\
    - Номер приемной комиссии\n- Как поступить в ВУЦ\n- Время приема ректората\
     - Информацию о заселении и медкомиссии\n"
        )
        sleep(2)
        await message.answer(
            f"Для получения инструкции о взаимодействии с\
            ассистентом нажмите /help\n\n\
    {hbold('Внимание, все сообщения логгируются!')}"
        )
\end{codepiece}

После получения сообщения сервером Telegram и передачей сообщения приложению
по протоколу HTTPS, что отражает нефункциональное требование к безопасности
SAF\_1, происходит перевод полученного текста в вектор и сопоставление
полученного вектора с векторами из созданного векторного пространства 
базы знаний. После поиска подходящих векторов происходит обратная трансформация
из векторного пространства в текстовое представление. 

На листинге 3.4 отображен процесс создания контекста и промт для
корректной работы языковой модели. Для работы системы выбираются 3 релевантных
вектора, так как это позволяет найти необходимые вектора без избыточности и
снизить нагрузку на сервис, и к тому же снизить расход токенов для LLM или
SLM\@.

\begin{codepiece}{in-place code}{Функция для поиска векторов и создания контекста}
    def augment_prompt(self, query: str, _embedings_storage=_embedings_storage):
        results = _embedings_storage(self).similarity_search(query, k=3)
        source_knowledge = "\n".join([x.page_content for x in results])
        augmented_prompt = f"""Используя предоставленный контекст ответь на следующий вопрос.

        Контекст:
        {source_knowledge}

        Вопрос: {query}"""
        return augmented_prompt
\end{codepiece}

Далее обогащенный текстовый запрос подается на вход языковой модели, после чего
проиходит генерация текстовых токенов декодером~\cite{a}. На листинге 3.5
отображен модуль, ответственный за генерацию текста. 

\begin{codepiece}{in-place code}{Функция для генерации ответа на заданный вопрос}
        def answer(self, message: Message) -> None:
            prompt = PromptTemplate(
                template="Ответь на поставленный вопрос: {subject}",
                input_variables=["subject"],
            )
    
            output = self.chat(
                [
                    SystemMessage(
                        content="""Ты профессиональный ассистент Российского
                        государственного гидрометеорологического университета,
                        который помогает студентам и преподавателям решать их
                        проблемы: навигация внутри университета, ответ на общие
                        вопросы, касающиеся РГГМУ. Если контекст большой, то
                        используй его по максимуму, но не теряй сути.
                        При ответе на вопросы общайся в деловом стиле и пиши
                        только по существу. """
                    ),
                    HumanMessage(
                        content=prompt.format(
                            subject=self.
                            augmentation.
                            augment_prompt(message)
                        )
                    ),
                ]
            )
  

            return output.content
\end{codepiece}

На рисунке~\ref{Текстовый ввод.png} можно наблюдать работу поиска и аугментации
внутри клиента Telegram.

\fig[0.7]{Текстовый ввод.png}{Работа приложения}

\subsect{Распознавание голоса}

Для пользователей предусмотренна возможность ввода как текстового сообщения,
так и управления ботом при помощи голоса или видео. Данная возможность
отражает нефункциональное требование NFR\_3. Для реализации данного требования
использовалась заранее обученная сверточная нейронная сеть из библиотеки
SpeechRecogintion. Для восприятия системой был написан модуль, сохраняющий
текстовое представление звука, после чего текст загружается в модуль,
отображенный на листинге 3.5.
На листинге 3.6 отображен модуль, ответственный за распознавание речи.

\begin{codepiece}{in-place code}{Функция для распознавания речи}
import speech_recognition as sr
import soundfile as sf
import os

def recognize_speech(audio_file):
    data, samplerate = sf.read(audio_file)
    sf.write('voice.wav', data, samplerate)

    recognizer = sr.Recognizer()

    with open('voice.wav', 'rb') as audio_file:
        with sr.AudioFile(audio_file) as source:
            audio_data = recognizer.record(source)
            recognizer.adjust_for_ambient_noise(source)

        try:
            text = recognizer.recognize_google(audio_data, language='ru-RU')
            os.remove('voice.wav')
            return text
        except sr.UnknownValueError:
            return "Попробуй ещё раз. Не могу разобрать, что ты говоришь."
        except sr.RequestError:
            return "Ну здесь наши полномочия все :(."
\end{codepiece}

Данное архитектурное решение позволяет уменьшить 
количество повторов при написании кода и разделить ответственность между
участками приложения.

Работу модуля распознавания голоса и преобразования его в текст можно увидеть
на рисунке~\ref{Speech2text.png}.

\fig[0.7]{Speech2text.png}{Преобразование голосового сообщения в текстовый запрос}

\subsect{Распознавание видео}

Так же для приложения был написан модуль для распознавания звука внутри
видеофайлов. Планируется доработать приложения, для поддержки управления 
глухонемыми людьми. В настоящий момент модуль распознавания видеофайлов
работает по аналогии с модулем распознавания аудиофайлов --- видео и аудио
дороги разделяются, после чего аудиодорожка распознается модулем распознавания
голосовых сообщений. Работа данного модуля приведена на изображении
~\ref{video2text.png}.

\fig{video2text.png}{Преобразование видеосообщения в текстовый запрос}

Отрывок кода, ответственный за разделение видеосообщения на звук и видео
приведен на листинге 3.7.

\begin{codepiece}{in-place code}{Разбиение видео и аудио}
    import moviepy.editor as mp
    import os
    def video_handler(video_note: str) -> None:
        clip = mp.VideoFileClip(video_note)
        clip.audio.write_audiofile("voice.wav")
        os.remove(video_note)
\end{codepiece}

\subsect{Промт-инженерия}

Для задания корректного поведения работы языковой модели и определения стиля
коммуникации с конечным пользователем были разработаны специальные
запросы-подсказки для LLM --- промты. Промты отвечают за многие аспекты работы
GPT-like моделей: 

\begin{itemize}
    \item Корректность вывода сообщений
    \item Стиль общения языковой модели
    \item Точность ответов
    \item Структура вывода ответа
\end{itemize}

Промт, написанный для модуля поиска релевантных векторов и обогащения ответа
можно наблюдать на листинге 3.8.

\begin{codepiece}{in-place code}{Промт для retrivial}
     augmented_prompt = f"""Используя предоставленный контекст ответь на следующий вопрос.

     Контекст:
     {source_knowledge}

     Вопрос: {query}"""
\end{codepiece}

Промт, написанный для задания поведения языковой модели можно наблюдать на 
листинге 3.9.

\begin{codepiece}{in-place code}{Поведенческий промт}
        prompt = PromptTemplate(
            template="Ответь на поставленный вопрос: {subject}",
            input_variables=["subject"],
        )

        output = self.chat(
            [
                SystemMessage(
                    content="""Ты профессиональный ассистент Российского
                    государственного гидрометеорологического университета,
                    который помогает студентам и преподавателям решать их
                    проблемы: навигация внутри университета, ответ на общие
                    вопросы, касающиеся РГГМУ. Если контекст большой, то
                    используй его по максимуму, но не теряй сути.
                    При ответе на вопросы общайся в деловом стиле и пиши
                    только по существу. """
                ),
                HumanMessage(
                    content=prompt.format(
                        subject=self.
                        augmentation.
                        augment_prompt(message)
\end{codepiece}

\subsect{Логирование}

Для отлавливания ошибок был реализован механизм логирования. Обработанные
системой сообщения сохраняются в следующем виде: уникальный идентификатор 
пользователя; имя пользователя; введенное пользователем сообщение и ответ,
полученный от системы. Данный механизм позволяет проверять корректность работы
системы, а при возникновении ошибки или попытке использовать систему не по
назначению можно идентифицировать пользователя и связаться с ним. Посмотреть
пример сохраненных логов можно на изображении~\ref{Сохраненные логи.png}.

\fig{Сохраненные логи.png}{Логированные сообщения}

\subsect{Тестирование}

Для проверки корректности приложения было проведено интеграционное
тестирование. Результаты тестирования были приведены в таблице~\ref{tbl:inttest}.
Работа приложения считается корректной при корректном ответе на более чем
$80\%$ вопросов.

\begin{longtbl}{inttest}
    {Результаты интеграционного тестирования}
    {N{4mm}|N{6cm}|N{10cm}}
        
    № & 
    \thead>{6cm}{Вопрос} & 
    \thead>{10cm}{Ответ}  \\\hline
    \endfirsthead

\caption*{Продолжение таблицы \thetable} \\
\hline
    № & 
    \thead>{6cm}{Вопрос} & 
    \thead>{10cm}{Ответ}  \\\hline
\endhead

    \mr{3}{1} & \mr{3}{Что ты умеешь?} & Я могу помочь вам найти информацию о
    направлениях подготовки в РГГМУ, предоставить примерные размеры стипендий 
    для разных категорий студентови ответить на другие вопросы, связанные
    с университетом.  \\\hline
    \mr{4}{2} & \mr{4}{Что нужно иметь для получения социальной стипендии?} & 
    Для получения социальной стипендии необходимо иметь следующие документы:
    паспорт, СНИЛС, ИНН и справки из соц.защиты. В деканате нужно написать 
    заявление, после чего либо самостоятельно отнести в бухгалтерию на 
    Воронежской 79, либо за вас это делает сам деканат.\\\hline
    \mr{4}{3} & \mr{4}{Как попасть на прием к Палкину} &
    Для записи на прием к Палкину Ивану Ивановичу необходимо зайти на сайт 
    РГГМУ и перейти по ссылке 
    ''https://forms.yandex.ru/u/65323419e010db2119cf9903/''. Там следует 
    выбрать удобное время для посещения и оставить свои 
    контактные данные.  \\\hline
    \mr{4}{4} & \mr{4}{Номер телефона студенческого отдела кадров} & 
    Номер телефона студенческого отдела кадров РГГМУ - 8(812) 409-13-49.  \\\hline
    \mr{4}{5} & \mr{4}{Как оформить бск?} & 
    БСК можно оформить разными способами: в «Центре изготовления льготных БСК», это возле ст.м. Площадь Александра Невского 2, если быть конкретнее, Чернорецкий переулок, д.3., также можно придти с готовой фотографией, паспортом и студенческим билетом в любую кассу метрополитена и оформить БСК там или заказать на сайте https://zakaz.ltkarta.ru/.  \\\hline
    \mr{5}{6} & \mr{5}{Расписание} & Расписание можно найти по данной ссылке: https://www.rshu.ru/university/stud/\\\hline
    \mr{4}{7} & \mr{4}{Направления, связанные с программированием} & 
    Направления, связанные с программированием, в РГГМУ: Прикладная информатика, Информационная безопасность телекоммуникационных систем, Бизнес-информатика.  \\\hline
    \mr{4}{8} & \mr{4}{Расскажи о РГГМУ} & РГГМУ - Российский государственный гидрометеорологический университет, основанный в 1930 году. Университет готовит специалистов в области гидрометеорологии, экологии, океанологии, гидрологии, метеорологии, геофизики, информационных систем и других смежных областях. В университете функционируют различные факультеты, включая метеорологический, гидрологический, экологический, морской права, экономики и управления. РГГМУ активно участвует в научных исследованиях и разработках, сотрудничает с международными организациями и имеет филиалы в различных регионах России. \\\hline
    \mr{5}{9} & \mr{5}{Исигт} & Институт информационных систем и геотехнологий (ИИСИГТ) является структурным подразделением Российского государственного гидрометеорологического университета. Он расположен по адресу: 195196, Россия, Санкт-Петербург, проспект Металлистов, дом 3. Директор института - Истомин Евгений Петрович, его электронная почта: dekanat\_is@rshu.ru.\\\hline
    \mr{4}{10} & \mr{4}{Как поступить в ВУЦ?} & 
    Для поступления в военный учебный центр (ВУЦ) РГГМУ необходимо выполнить следующие шаги:

1. Прибыть в ВУЦ после соответствующего объявления на сайте и подать заявление на имя ректора РГГМУ. Образец заявления можно скачать на сайте или заполнить в ВУЦ. К заявлению нужно приложить копии документов, включая согласие на обработку персональных данных, автобиографию, эссе "Почему я выбираю военную подготовку", удостоверение гражданина Российской Федерации, подлежащего призыву в ВС РФ, паспорт, документ о временной регистрации (для иногородних), зачетную книжку, анкету на бумажном и электронном носителе. Также потребуются фотографии 3x4 и 4x6.

2. Получить в ВУЦ бланк медицинской карты и направление для прохождения медицинской комиссии. Подготовить документы, необходимые для прохождения медицинской комиссии, согласно перечня (уточняется самостоятельно в военном комиссариате).

3. В марте-апреле пройти предварительный отбор в военном комиссариате по месту регистрации (или пребывания для иногородних) и сдать его результаты в учебную часть ВУЦ.

4. До 1 мая представить в ВУЦ результаты предварительного отбора, медицинскую карту и результаты профессионального психологического отбора, характеристику из деканата, документы, подтверждающие преимущественное право при проведении конкурсного отбора.

5. В апреле-мае сдать нормативы по физической подготовке на кафедре физического воспитания.

6. Конкурсная комиссия состоится в июле. В случае успешного прохождения по конкурсу до 1 августа заключить договор установленной формы.

7. До 28 августа приобрести специальную форму.

8. 1 сентября 2024 года приступить к занятиям согласно расписанию.

Списки студентов, прошедших конкурсный отбор, будут размещены на стенде в ВУЦ.\\\hline
    \mr{4}{11} & \mr{4}{Когда приходит стипендия и что влияет на повышенную стипендию?} & 
    Стипендия приходит с 20-25 числа каждого месяца. Повышенная стипендия зависит от успеваемости студента. Она может быть повышенной за хорошую (одна четверка в сессии) или отличную (все "5") успеваемость. Также на нее могут влиять активное участие в жизни университета, научная, спортивная и культурно-творческая деятельность. \\\hline
    \mr{5}{12} & \mr{5}{Где можно забрать аттестат?} & Аттестат можно забрать в 3 корпусе РГГМУ - Воронежская 79, кабинет 16/3(первый этаж) - Студенческий отдел кадров.  \\

\end{longtbl}

Тестирование проводилось вручную. Также проводилось нагрузочное тестирование:
сервис был опубликован перед студентами, после чего возник массовый наплыв
пользователей. Сервис выдержал нагрузку и корректно работал. Пиковая нагрузка
--- 10 пользователей. Ни один из модулей не отказал. Работа в многопоточном
режиме показала себя на достойном уровне.

Так же во время тестирования проверялась скорость поиска и генерации ответа.
Максимальное время на генерацию ответа --- 12,68 секунд, а минимальное --- 4,33.
В среднем поиск релевантных векторов, агументация промта и генерация ответа
занимают$~\approx6$ секунд, что соответствует нефункциональному требованию к
производительности сервиса PER\_1.

Так же по результатам тестирования можно отметить соотвествие ответов системы
нефункциональным требованиям NFR\_4 и NFR\_5, опираясь на строки 7 и 8 таблицы
~\ref{tbl:inttest}.

\structel{ЗАКЛЮЧЕНИЕ}

В процессе работы были выполнены поставленные задачи:

\begin{enumerate}
    \item Проведен системный анализ
    \item Произведена оценка экономической эффективности
    \item Проектирование системы
    \item Разработка системы
    \item Тестирование системы
\end{enumerate}

В результате выполнения
задач мой было реализовано приложение, соответствующее всем функциональным и
нефункциональным требованиям.

Созданный бот можно найти по адресу \emph{https://t.me/IISIGT\_bot} или по логину 
\emph{@IISIGT\_bot} внутри клиента Telegram.

\showbib

\annex{Диаграмма последовательности}

\begin{figure}[H]
    \centering
    \includegraphics[width=1.2\textwidth, angle=90]{Диаграмма последовательности.png}
    \caption{Диаграмма последовательности}\label{Диаграмма последовательности.png}
\end{figure}

\annex{Конфигурация NeoVim}

\begin{codepiece}{in-place code}{Отрывок конфигурации языкового сервера для работы NeoVim}
local lspconfig = require 'lspconfig'

local capabilities = require('cmp_nvim_lsp').default_capabilities()

lspconfig.pylsp.setup {
  capabilities = capabilities,
  settings = { pylsp = { plugins = require('project.config').pylsp_plugins } },
}
lspconfig.tsserver.setup {
  capabilities = capabilities,
}
lspconfig.ccls.setup {
  capabilities = capabilities,
}
lspconfig.gopls.setup {
  capabilities = capabilities
}

-- `:help vim.diagnostic.*`
vim.keymap.set('n', '<space>e', vim.diagnostic.open_float)
\end{codepiece}

\annex{Диаграмма Ганта}

\begin{figure}[H]
    \centering
    \includegraphics[width=1.2\textwidth, angle=90]{Диаграмма Ганта.png}
    \caption{Диаграмма Ганта}\label{Диаграмма Ганта.png}
\end{figure}

\annex{Конфигурация \LaTeX}

\begin{codepiece}{in-place code}{Конфигурация \LaTeX для титульного листа по ГОСТ 7.32}
\begin{titlepage}
    \singlespacing
    \setlength{\parindent}{0pt}
    \begin{center}
        Министерство науки и высшего образования Российской Федерации\\
        Фереральное государственное бюджетное образовательное учреждение
высшего образования\\
        Российский государственный гидрометеорологический университет\\
        (РГГМУ)\\
        Институт информационных систем и геотехнологий\\
        Направление подготовки: 09.03.03 <<Прикладная информатика>>\\
        Профиль подготовки: <<Прикладные информационные системы\
        и геотехнологии>>

    \end{center}
    \vspace{\oneinterv}
    \begin{center}
    ВЫПУСКНАЯ КВАЛИФИКАЦИОННАЯ РАБОТА\\
    (БАКАЛАВРСКАЯ РАБОТА)
    \end{center}
    \vspace{\oneinterv}
    На тему: <<Разработка приложения интеллектуального ассистента на базе
    технологий глубокого обучения.>>
    \vspace{\twointerv}

    \vfill

    \begin{tabular}{N{70mm}N{80mm}}
        Научный руководитель,\\
\end{codepiece}

\annex{Реализация обработки различных данных}

\begin{codepiece}{in-place code}{Обработка текста}
    @dp.message(F.text)
    async def chat_handler(message: Message) -> None:
        try:
            msg = await message.answer(dummy_message)
            talkbox_msg = talkbox.answer(message.text)
            await message.answer(f"{talkbox_msg}\n\n{stub}")
            await msg.delete()
            with open(logs, "a") as log:
                log_msg = f"ID: {message.from_user.id} | Username: {message.from_user.full_name} | Message: {message.text} | Answer: {talkbox_msg}\n\n"
                log.write(log_msg)
                print(log_msg)

        except TypeError:
            await message.answer(
                "Что-то пошло не так! Сообщил разработчку о произошедшей неполадке!"
            )
            with open(logs, "a") as log:
                log.write(
                    f"ID: {message.from_user.id} | Username: {message.from_user.full_name} | ErrorMessage\n\n"
                )


\end{codepiece}

\begin{codepiece}{in-place code}{Обработка звука}
@dp.message(F.voice)
async def audio_handler(message: Message) -> None:
    voice_message = await bot.get_file(message.voice.file_id)
    voice_message = voice_message.file_path
    await bot.download_file(voice_message, "voice.wav")
    with open("voice.wav", "rb") as voice_message:
        text = recognize_speech(voice_message)
        msg = await message.answer(dummy_message)
        talkbox_msg = talkbox.answer(text)
        await message.answer(f"{talkbox.
            answer(text)}\n\n{stub}")
        await msg.delete()
        with open(logs, "a") as log:
            log_msg = f"ID: {message.from_user.id} | Username: {message.from_user.full_name} | Message: {text} | Answer: {talkbox_msg}\n\n"
            log.write(log_msg)
\end{codepiece}

\begin{codepiece}{in-place code}{Обработка видео}
@dp.message(F.video_note)
async def videomessage_handler(message: Message) -> None:
    voice_message = await bot.get_file(message.video_note.file_id)
    voice_message = voice_message.file_path
    await bot.download_file(voice_message, video_name)
    video_handler(video_name)
    with open("voice.wav", "rb") as voice_message:
        text = recognize_speech(voice_message)
        msg = await message.answer(dummy_message)
        talkbox_msg = talkbox.answer(text)
        await message.answer(f"{talkbox_msg}\n\n{stub}")
        await msg.delete()
        with open(logs, "a") as log:
            log_msg = f"ID: {message.from_user.id} | Username: {message.from_user.full_name} | Message: {text} | Answer: {talkbox_msg}\n\n"
            log.write(log_msg)

\end{codepiece}

\end{document}
