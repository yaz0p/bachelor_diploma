\documentclass[a4paper,14pt]{extarticle}

% кодировка
\usepackage[utf8]{inputenc}
\usepackage[T2A]{fontenc}

% поля
\usepackage[left=30mm,right=15mm,top=20mm,bottom=20mm]{geometry}

% переносы слов
\usepackage[english,main=russian]{babel}

% шрифт Таймс
\usepackage{tempora}
%\usepackage{newtxmath}
%
% межстрочный интервал
\usepackage[onehalfspacing]{setspace}

% отступ первой строки
\usepackage{indentfirst}
\setlength{\parindent}{1.25cm}

% скрытый структурный элемент
\newcommand{\hidedstructel}[1]{%
    \clearpage
    \section*{#1}%
}

% структурный элемент
\newcommand{\structel}[1]{%
    \hidedstructel{#1}
    \addcontentsline{toc}{section}{#1}%
}

% счетчик приложений
\usepackage{totcount}
\newtotcounter{annexcount}

% приложение
\renewcommand{\thesection}{\Asbuk{section}}
\newcommand{\annex}[1]{%
    \stepcounter{annexcount}%
    \clearpage
    \setcounter{figure}{0}  % сбросить нумерацию внутри раздела
    \setcounter{table}{0}%
    \setcounter{listing}{0}%
    \renewcommand{\thetable}{\thesection.\arabic{table}}
    \renewcommand{\thefigure}{\thesection.\arabic{figure}}
    \renewcommand{\thelisting}{\thesection.\arabic{listing}}
    \section{#1}%
}

% оформление структурного элемента и приложения
\usepackage{titlesec}
\titleformat{\section}
    [display]                   % форма
    {\filcenter\bfseries}       % формат полностью
    {ПРИЛОЖЕНИЕ \thesection}    % метка
    {}                          % отступ от метки
    {}                          % код перед телом

% раздел
\newcommand{\sect}[1]{%
    \clearpage
    \setcounter{figure}{0}  % сбросить нумерацию внутри раздела
    \setcounter{table}{0}
    \setcounter{listing}{0}
    \subsection{#1}
    \renewcommand{\theparagraph}{\thesubsection.\arabic{paragraph}}%
}
\titleformat{\subsection}{\filright\bfseries}{}{}{\thesubsection\hspace{1em}}
\titlespacing*{\subsection}
    {\parindent}    % отступ слева
    {1.5em}              % сверху
    {1.5em}              % снизу
\renewcommand{\thesubsection}{\arabic{subsection}}

% подраздел
\usepackage{placeins}
\newcommand{\subsect}[1]{%
    \FloatBarrier
    \subsubsection{#1}
    \renewcommand{\theparagraph}{\thesubsubsection.\arabic{paragraph}}
}
\titleformat{\subsubsection}{\filright\bfseries}{}{}{\thesubsubsection\hspace{1em}}
\titlespacing*{\subsubsection}
    {\parindent}
    {1.5em}
    {1.5em}

% пункт
\newcommand{\parag}{
    \paragraph{}
}
\titleformat{\paragraph}[runin]{}{\theparagraph}{1em}{ }  % для отступа
\titlespacing*{\paragraph}{\parindent}{}{}

% подпункт
\newcommand{\subparag}{
    \subparagraph{}
}
\titleformat{\subparagraph}[runin]{}{\thesubparagraph}{1em}{ }
\titlespacing*{\subparagraph}{\parindent}{}{}

% содержание
\usepackage{etoc}
\setcounter{tocdepth}{3}

% глубина нумерации разделов
\setcounter{secnumdepth}{5}

% перечисления
\usepackage{enumitem}
\setlist{
    topsep=0,                   % отступ сверху и снизу списка
    partopsep=0,                % то же самое
    leftmargin=0,               % отступ слева
    labelsep=0,                 % отступ метки
    align=left,                 % выравнивание метки
    listparindent=\parindent,   % отступ первой строки абзаца
    itemsep=0,                  % отступ между элементами
    parsep=0                    % отступ между абзацами и элементами
}
\setlist[itemize]{
    label=--~,  % в списках тире короткое, в тексте - длинное
    labelwidth=1.2em,
    itemindent=\parindent+\labelwidth
}
\setlist[enumerate]{
    label=\arabic*),
    labelwidth=1.4em,
    itemindent=\parindent+\labelwidth
}

% перечисление с буквенными метками
\AddEnumerateCounter*{\asbuk}{\c@asbuk}{7}
\newlist{asblist}{enumerate}{2}
\setlist[asblist]{
    label=\asbuk*),
    labelwidth=1.4em,
    itemindent=\parindent+\labelwidth
}

% подписи
\usepackage[singlelinecheck=false]{caption}
\DeclareCaptionLabelSeparator{gost}{~---~}
\captionsetup{labelsep=gost}

% иллюстрация
\newcommand{\fig}[3][1]{
    \begin{figure}[H]
        \centering
        \includegraphics[width=#1\textwidth]{#2}
        \caption{#3}\label{#2}
    \end{figure}
}
\renewcommand{\thefigure}{\thesubsection.\arabic{figure}}
\DeclareCaptionLabelFormat{gostfigure}{Рисунок #2}
\captionsetup[figure]{justification=centering, labelformat=gostfigure, position=bottom}
% font=singlespacing по умолчанию
%skip=-6pt

% листинг
\usepackage[newfloat, cache=false]{minted}
\newcommand{\lst}[2]{
    \begin{listing}[H]
        \centering
        \caption{#2}\label{#1}
        \begin{minipage}[t]{.8\textwidth}
            \inputminted[
                fontsize=\small,
                frame=single,
                breaklines,
                linenos
            ]{text}{#1}
        \end{minipage}
    \end{listing}
}
\renewcommand{\thelisting}{\thesubsection.\arabic{listing}}
\DeclareCaptionLabelFormat{custlisting}{Листинг #2}
\captionsetup[listing]{justification=raggedright, labelformat=custlisting, position=top}

% размер номера строки
\renewcommand{\theFancyVerbLine}{\rmfamily{\small \oldstylenums{\arabic{FancyVerbLine}}}}

% код в документе
\newenvironment{codepiece}[2]
{
    \VerbatimEnvironment
    \begin{listing}[H]
        \centering
        \caption{#2}\label{lst:#1}
        \begin{minipage}[t]{.8\textwidth}
            \begin{minted}[
                fontsize=\small,
                frame=single,
                breaklines,
                linenos
            ]{text}
}{
            \end{minted}
        \end{minipage}
    \end{listing}
}

% таблица
\newenvironment{tbl}[3]
{
    \begin{table}[H]
        \small
        \centering
        \caption{#2}\label{tbl:#1}
        \begin{tabular}{|#3|}
            \hline
}{
            \hline
        \end{tabular}
    \end{table}
}
\renewcommand{\thetable}{\thesubsection.\arabic{table}}
\DeclareCaptionLabelFormat{gosttable}{Таблица #2}
\captionsetup[table]{justification=raggedright, labelformat=gosttable, position=top}

\usepackage{tabularx}

% объединение строк
\usepackage{multirow}
\newcommand{\mr}[2]{\multirow[t]{#1}{=}{#2}}

% колонки
\usepackage{array}
\newcolumntype{M}[1]{>{\centering\arraybackslash}m{#1}}
\newcolumntype{N}[1]{>{\raggedright\arraybackslash}p{#1}}

% заголовок таблицы
\usepackage{xparse}
\NewExpandableDocumentCommand\thead{t< t> O{1} m m}{%
    \IfBooleanTF{#1}{%
        \IfBooleanTF{#2}{%
            \multicolumn{#3}{|M{#4}|}{#5}%
        }{%
            \multicolumn{#3}{|M{#4}}{#5}%
        }
    }{%
        \IfBooleanTF{#2}{%
            \multicolumn{#3}{M{#4}|}{#5}%
        }{%
            \multicolumn{#3}{M{#4}}{#5}%
        }%
    }%
}

% код в таблице
\newenvironment{tabcode}[1]
{
    \VerbatimEnvironment
    \begin{minipage}[t]{#1\textwidth}
    \begin{minted}[fontsize=\small, breaklines]{text}
}{
    \end{minted}
    \end{minipage}
}

% длинная таблица
\usepackage{longtable}
\newenvironment{longtbl}[3]
{
    \small
    \begin{longtable}{|#3|}
        \caption{#2}\label{tbl:#1}\\
        \hline
}{
        \hline
    \end{longtable}
}

% математика
\usepackage{mathtools}  % amsmath
\numberwithin{equation}{subsection}

% графики
\usepackage{tikz, pgfplots}
\pgfplotsset{compat=newest}

\usepackage{csquotes}
\usepackage{adjustbox}
\usepackage{float}
\usepackage{url}

% источники
\usepackage[%
    backend=biber,%
    bibstyle=gost-numeric%
]{biblatex}
\addbibresource{bibliography.bib}
\newcommand{\showbib}{%
    \structel{СПИСОК ИСПОЛЬЗОВАННЫХ ИСТОЧНИКОВ}%
    \printbibliography[heading=none]%
}

% отступы в источниках
\defbibenvironment{bibliography}
    {\list
        {}
        {\setlength{\leftmargin}{0}%
         \setlength{\itemindent}{\parindent}%
         \setlength{\itemsep}{0}%
         \setlength{\parsep}{0}}}
    {\endlist}
    {\item
     \printtext[labelnumberwidth]{%
        \printfield{labelprefix}%
        \printfield{labelnumber}%
     }%
     \hspace{0.5em}}

% метка без точки
\DeclareFieldFormat{labelnumberwidth}{#1}

% номер последней страницы
\usepackage{lastpage}

% счетчик источников
\newtotcounter{bibcount}
\AtEveryBibitem{
    \stepcounter{bibcount}%
}

% счетчики таблиц и рисунков
\usepackage{xassoccnt}
\newtotcounter{tblcount}
\DeclareAssociatedCounters{table}{tblcount}
\newtotcounter{figcount}
\DeclareAssociatedCounters{figure}{figcount}

%% для отладки
%%\usepackage{showframe}
%%\renewcommand\ShowFrameLinethickness{0.25pt}
%%\renewcommand*\ShowFrameColor{\color{red}}
%%\usepackage{graphicx}

\begin{document}
\newcommand{\signplace}{\underline{\hspace{40mm}}}
\newcommand{\dateblank}{%
    <<\underline{\hspace{10mm}}>> \underline{\hspace{30mm}} 2024 г.%
}
\newlength{\twointerv}\setlength{\twointerv}{28.34pt}

\begin{titlepage}
    \singlespacing
    \setlength{\parindent}{0pt}
    \begin{center}
        Министерство науки и высшего образования Российской Федерации\\
        Фереральное государственное бюджетное образовательное учреждение
высшего образования\\
        Российский государственный гидрометеорологический университет\\
        (РГГМУ)\\
        Институт информационных систем и геотехнологий\\
        Направление подготовки: 09.03.03 <<Прикладная информатика>>\\
        Профиль подготовки: <<Прикладные информационные системы\
        и геотехнологии>>

    \end{center}
    \vspace{\oneinterv}
    \begin{center}
    ВЫПУСКНАЯ КВАЛИФИКАЦИОННАЯ РАБОТА\\
    (БАКАЛАВРСКАЯ РАБОТА)
    \end{center}
    \vspace{\oneinterv}
    На тему: <<Разработка приложения интеллектуального ассистента на базе
    технологий глубокого обучения.>>
    \vspace{\twointerv}

    \vfill

    \begin{tabular}{N{70mm}N{80mm}}
        Научный руководитель,\\
        к.т.н Петров Я.А.& \signplace{}\\
        \vspace{5mm}
        Исполнитель студент группы ПИ-Б20-2-2,\\Попов В.Н. & \signplace{}\\
        \vspace{5mm}
        <<К защите допускаю>>,\\
        Заведующий кафедрой,\\
        к.т.н,\\
        Колбина О.Н.\\
        \dateblank{} & \signplace{}\\
    \end{tabular}

    \vfill

    \begin{center}
        Санкт-Петербург 2024 г.
    \end{center}
\end{titlepage}
\setcounter{page}{2}
\pagestyle{}


\structel{АННОТАЦИЯ}
Бакалаврская работа "кол-во страниц", "кол-во рисунков", "кол-во таблиц",
"кол-во источников", "кол-во приложений"

MASHINE LEARNING, DEEP LEARNING, NATURAL LANGUAGE PROGRAMMING, 
ARTIFICIAL INTELLIGENCE

Объектом исследования являются большие языковые модели (LLM).

Цель работы --- проектирование и разработка приложения, которое позволяет
взаимодействовать с структурой предприятий посредством приложения для 
коммуникации, которое использует LLM.

В результате работы реализован ассистент для высшего учебного заведения
на базе искусственного интеллекта, позволяющий получать расписание занятий, 
информацию о ВУЗе и справочную информацию.

Созданный ассистент может быть легко интегрирован в другие образовательные
организации, медицинские учреждения, также применяться бизнесом для своих нужд.

Ассистент позволяет использовать неявный поиск и искать всю информацию в одном
месте, что позволяет оптимизировать некоторые процессы, позитивно сказывается на 
пользовательском опыте при коммуникации с инфраструктурой предприятия 
и может оказать влияние на репутацию предприятия.

\begingroup

\parindent 0pt
\newlength{\pagewidth}\setlength{\pagewidth}{1.1em}

\newlength{\sectnum}\setlength{\sectnum}{8.4em}
\newlength{\ssectnum}\setlength{\ssectnum}{1em}
\newlength{\sssectnum}\setlength{\sssectnum}{2em}

\newlength{\sssectindent}\setlength{\sssectindent}{2em}

\newcommand*{\entrybody}{%
    \raggedright%
    \etocname\nobreak%
    \leaders\etoctoclineleaders\hfill%
    \rlap{\makebox[\pagewidth][r]{\etocpage}}%
    \vspace{0.56em}% хак для отступа
}

\etocsetstyle{section}
    {}
    {\leavevmode\etocifnumbered{\leftskip \sectnum}{\leftskip 0}}
    {\normalsize\etocifnumbered%
        {\llap{\makebox[\sectnum][l]{ПРИЛОЖЕНИЕ \etocnumber}}%
            \parbox[t][][t]{\textwidth-\sectnum-\pagewidth}{\entrybody}}%
        {\parbox[t][][t]{\textwidth-\pagewidth}{\entrybody}}\par}
    {}
\etocsetstyle{subsection}
    {}
    {\leavevmode\leftskip \ssectnum}
    {\normalsize\llap{\makebox[\ssectnum][l]{\etocnumber}}%
        \parbox[t][][t]{\textwidth-\ssectnum-\pagewidth}{\entrybody}\par}
    {}
\etocsetstyle{subsubsection}
    {}
    {\leavevmode\setlength{\leftskip}{\sssectnum+\sssectindent}\relax}
    {\normalsize\llap{\makebox[\sssectnum][l]{\etocnumber}}%
        \parbox[t][][t]{\textwidth-\sssectnum-\pagewidth-\sssectindent}{\entrybody}\par}
    {}

\etocsettocstyle{\hidedstructel{СОДЕРЖАНИЕ}}{}

\tableofcontents

\endgroup


\structel{ВВЕДЕНИЕ}
За последние десятилетие произошел огромный скачок в автоматизации многих процессов:
рекомендательные системы, A/B тестирование, антифрод системы, анализ и обработка
больших объемов информации. Одним из таких инструментов являются большие
языковые модели.

В данной выпускной квалификационной работе будет разработана информационная
система, использующая LLM для генерации ответов широкий список вопросов.

В качестве интерфейса для пользователя была выбрана оболочка в виде чат-бота.
Чат-боты давно вошли в жизнь большинства населения. Это подтверждается 
информацией аналитической компании <<eMarketer>>, согласно которой, чат-ботами
пользуются более 1,4 млрд. человек на планете.

Для выполнения поставленной цели были поставлены следующие задачи:
\begin{itemize}
    \item Выполнить анализ предметной области;
    \item Провести сравнительный анализ информационных систем;
    \item Изучить сроки реализации проекта;
    \item Смоделировать схему бизнес-процессов;
    \item Составить описание документов бизнес-процессов;
    \item Сформировать перечень требований к ИС;
    \item Исследовать подходы SWOT
    \item Описать сценарии вариантов использования;
    \item Визуализировать описанные сценарии вариантов использования;
    \item Создать модель диаграммы компонентов;
    \item Создать модель диаграммы развертывания;
    \item Реализовать бизнес-логику ассистента и перенести его в интерфейс бота;
\end{itemize}

В работе будет рассматриваться РГГМУ (далее Университет), но применяться бот
сможет не только в конкретном учебном заведении, а для любых предприятий.

Во время разработки ассистента использовалась методология Agile. Она позволила
работать в удобном темпе и формировать требования во время разработки.

\sect{Предпроектный анализ}
\subsect{Анализ предметной области}



\subsect{Сравнительный анализ}
\subsect{Системный анализ}

\sect{Проектирование информационной системы}
\subsect{Концептуальное проектирование}
\subsect{Функциональное проектирование}
\subsect{Диаграмма последовательности}
\subsect{Диаграмма развертывания}

\sect{Разработка системы}
\subsect{Выбор средств разработки}
\subsect{Что-то про структуру приложения}
\subsect{Что-то про извлечение сущности из фраз пользователя и их метчинг с сущностями из базы знаний}
\subsect{Что-то про метод ближайших соседей и обогащение ответа}
\subsect{Что-то про генерацию ответа}

\end{document}
