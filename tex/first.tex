\input{./inputs/preamble}
\begin{document}
\newcommand{\signplace}{\underline{\hspace{40mm}}}
\newcommand{\dateblank}{%
    <<\underline{\hspace{10mm}}>> \underline{\hspace{30mm}} 2024 г.%
}
\newlength{\twointerv}\setlength{\twointerv}{28.34pt}

\begin{titlepage}
    \singlespacing
    \setlength{\parindent}{0pt}
    \begin{center}
        Министерство науки и высшего образования Российской Федерации\\
        Фереральное государственное бюджетное образовательное учреждение
высшего образования\\
        Российский государственный гидрометеорологический университет\\
        (РГГМУ)\\
        Институт информационных систем и геотехнологий\\
        Направление подготовки: 09.03.03 <<Прикладная информатика>>\\
        Профиль подготовки: <<Прикладные информационные системы\
        и геотехнологии>>

    \end{center}
    \vspace{\oneinterv}
    \begin{center}
    ВЫПУСКНАЯ КВАЛИФИКАЦИОННАЯ РАБОТА\\
    (БАКАЛАВРСКАЯ РАБОТА)
    \end{center}
    \vspace{\oneinterv}
    На тему: <<Разработка приложения интеллектуального ассистента на базе
    технологий глубокого обучения.>>
    \vspace{\twointerv}

    \vfill

    \begin{tabular}{N{70mm}N{80mm}}
        Научный руководитель,\\
        к.т.н & \signplace{}Петров Я.А.\\
        \vspace{5mm}
        Исполнитель,\\студент группы ПИ-Б20-2-2 & \signplace{} Попов В.Н.
    \end{tabular}

    \vfill

    \begin{center}
        Санкт-Петербург 2024 г.
    \end{center}
\end{titlepage}
\setcounter{page}{2}
\pagestyle{}


\structel{АННОТАЦИЯ}
Бакалаврская работа "кол-во страниц", "кол-во рисунков", "кол-во таблиц",
"кол-во источников", "кол-во приложений"

MASHINE LEARNING, DEEP LEARNING, NATURAL LANGUAGE PROGRAMMING, 
ARTIFICIAL INTELLIGENCE

Объектом исследования являются большие языковые модели (LLM).

Цель работы --- проектирование и разработка приложения, которое позволяет
взаимодействовать с структурой предприятий посредством приложения для 
коммуникации, которое использует LLM.

В результате работы реализован ассистент для высшего учебного заведения
на базе искусственного интеллекта, позволяющий получать расписание занятий, 
информацию о ВУЗе и справочную информацию.

Созданный ассистент может быть легко интегрирован в другие образовательные
организации, медицинские учреждения, также применяться бизнесом для своих нужд.

Ассистент позволяет использовать неявный поиск и искать всю информацию в одном
месте, что позволяет оптимизировать некоторые процессы, позитивно сказывается на 
пользовательском опыте при коммуникации с инфраструктурой предприятия 
и может оказать влияние на репутацию предприятия.

\input{./inputs/toc}

\structel{ВВЕДЕНИЕ}
За последние десятилетие произошел огромный скачок в автоматизации многих процессов:
рекомендательные системы, A/B тестирование, антифрод системы, анализ и обработка
больших объемов информации. Одним из таких инструментов являются большие
языковые модели.

В данной выпускной квалификационной работе будет разработана информационная
система, использующая LLM для генерации ответов широкий список вопросов.

В качестве интерфейса для пользователя была выбрана оболочка в виде чат-бота.
Чат-боты давно вошли в жизнь большинства населения. Это подтверждается 
информацией аналитической компании <<eMarketer>>, согласно которой, чат-ботами
пользуются более 1,4 млрд. человек на планете.

Для выполнения поставленной цели были поставлены следующие задачи:
\begin{itemize}
    \item Выполнить анализ предметной области;
    \item Провести сравнительный анализ информационных систем;
    \item Изучить сроки реализации проекта;
    \item Смоделировать схему бизнес-процессов;
    \item Составить описание документов бизнес-процессов;
    \item Сформировать перечень требований к ИС;
    \item Исследовать подходы SWOT
    \item Описать сценарии вариантов использования;
    \item Визуализировать описанные сценарии вариантов использования;
    \item Создать модель диаграммы компонентов;
    \item Создать модель диаграммы развертывания;
    \item Реализовать бизнес-логику ассистента и перенести его в интерфейс бота;
\end{itemize}

В работе будет рассматриваться РГГМУ (далее Университет), но применяться бот
сможет не только в конкретном учебном заведении, а для любых предприятий.

Во время разработки ассистента использовалась методология Agile. Она позволила
работать в удобном темпе и формировать требования во время разработки.

\sect{Предпроектный анализ}
\subsect{Анализ предметной области}



\subsect{Сравнительный анализ}
\subsect{Системный анализ}

\sect{Проектирование информационной системы}
\subsect{Концептуальное проектирование}
\subsect{Функциональное проектирование}
\subsect{Диаграмма последовательности}
\subsect{Диаграмма развертывания}

\sect{Разработка системы}
\subsect{Выбор средств разработки}
\subsect{Что-то про структуру приложения}
\subsect{Что-то про извлечение сущности из фраз пользователя и их метчинг с сущностями из базы знаний}
\subsect{Что-то про метод ближайших соседей и обогащение ответа}
\subsect{Что-то про генерацию ответа}

\end{document}
