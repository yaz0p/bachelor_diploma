\input{../inputs/preamble}
\begin{document}

\subsect{INTRO}
Уважаемый председатель государственной экзаменационной комиссии, уважаемые
члены государственной экзаменационной комиссии и уважаемые присутствующие!
Позвольте представить Вашему вниманию мою выпускную квалификационную работу

\subsect{Актуальность}
Актуальность моего приложения обусловлена множеством аспектов, но ключевые:
возрастающее количество пользователей языковых моделей для бизнесом
(мне было интересно собственноручно создать такой проект), в свою очередь
актуальность для ВУЗа --- создание универсального ассистента, способного
отвечать на заданые вопросы без хардкодинга и с возможностью неявного поиска.

\subsect{Цели и задачи}

Передо мной была поставлена следующая цель: спроектировать и разработать
приложение для взаимодействия с структурой предприятия посредством какого-либо
интерфейса. Был выбран интерфейс в виде телеграмм бота, но мою систему можно
без особых сложностей подключить как к ВКонтакте или Яндекс.Алисе. В настоящий
момент работает только в Telegram.

Поставленные задачи отображены на слайде.

\subsect{Сравнительный анализ}

Во время начала разработки моей системы непосредственных конкурентов у 
моей системы не было, поэтому я сравнивал свою систему с похожими решениями.
Недавно начало разрабатываться подобное решение от компании Сбербанк.

\subsect{Swot}

Для выявления сильных строн и потенциальных угроз был проведен SWOT анализ.

\subsect{Временные оценки}

На данном сладе расположена оценка временных затрат, а так же затрат на
разработку. 

\subsect{Гант}

На данном сладе расположена диаграмма ганта с визуализацией этапов разработки

\subsect{Функциональное моделирование}

На данном слайде схематично отображены функциональные требования.
Таким образом система должна уметь: приветствовать пользователя, уметь
объяснить пользователю как управлять системой, а так же предоставлять
информацию о учреждении.

\subsect{Схема работы приложения}

Пользователь отправляет запрос сервису, после чего запрос преобразуется в
вектор. Для преобразования вектора испольуется малослойная нейронная сеть.
Далее полученный вектор сравнивается при помощи вычисления косинусного и 
Евклидового растояний. Вектора с наибольшим коэфициентом подобия достаются
из базы знаний и подставляются в промт. Далее промт с данными из базы знаний
и запросом от пользователя отправляется на вход языковой модели, после чего
происходит генерация ответа, основываясь на предоставленном контексте.
После окончания генерации ответа происходит отправка ответа пользователю.

\subsect{Используемые технологии}

В основном использовалось три технологии: Python для разработки сервиса и RAG,
PyTorch для глубокого обучения, в виду того, что многие методы удобно 
написаны на языке C++, который во много раз быстрее Python, а так же Docker
для упаковки приложения в контейнер для сборки всех необходимых зависмостей,
что позволяет запустить приложение на любом устройстве с любой операционной
системой.

\subsect{Векторизация}

Векторизация происходит при помощи малослойной нейронной сети. На вход подается
текстовый корпус. Для оптимизации используется градиентный спуск.

\subsect{Сопоставление векторов}

Сопоставление вектор происходит при помощи расчета Евклидового расстояния между
векторами. На слайде можно наблюдать визуализацию данного метода для K=5. 
В моей системе используется K=3, чтобы уместиться в контекстное окно 
языковой модели.

\subsect{Расчет экономической эффективности}

Для расчета экономической эффективности использовалась предоставленная на
слайде формула. Процент окупаемости составил 300 процентов, что я считаю
довольно хорошим результатом.

\subsect{Реализация ИС}

На слайдах можно наблюдать реализацию системы. Работа приложения отвечает
описанным функциональным требованиям.
Так же планировался блок для сурдо-перевода, но у меня не было такого
количества данных для тренировки модели + это очень трудозатратно.
Посмотреть работы приложения можно отсканировав QR-код. Приложение работает
на моем персональном компьютере и возможно у меня кончится видеопамять)

\end{document}
