\documentclass[a4paper,14pt]{extarticle}

% кодировка
\usepackage[utf8]{inputenc}
\usepackage[T2A]{fontenc}

% поля
\usepackage[left=30mm,right=15mm,top=20mm,bottom=20mm]{geometry}

% переносы слов
\usepackage[english,main=russian]{babel}

% шрифт Таймс
\usepackage{tempora}
%\usepackage{newtxmath}
%
% межстрочный интервал
\usepackage[onehalfspacing]{setspace}

% отступ первой строки
\usepackage{indentfirst}
\setlength{\parindent}{1.25cm}

% скрытый структурный элемент
\newcommand{\hidedstructel}[1]{%
    \clearpage
    \section*{#1}%
}

% структурный элемент
\newcommand{\structel}[1]{%
    \hidedstructel{#1}
    \addcontentsline{toc}{section}{#1}%
}

% счетчик приложений
\usepackage{totcount}
\newtotcounter{annexcount}

% приложение
\renewcommand{\thesection}{\Asbuk{section}}
\newcommand{\annex}[1]{%
    \stepcounter{annexcount}%
    \clearpage
    \setcounter{figure}{0}  % сбросить нумерацию внутри раздела
    \setcounter{table}{0}%
    \setcounter{listing}{0}%
    \renewcommand{\thetable}{\thesection.\arabic{table}}
    \renewcommand{\thefigure}{\thesection.\arabic{figure}}
    \renewcommand{\thelisting}{\thesection.\arabic{listing}}
    \section{#1}%
}

% оформление структурного элемента и приложения
\usepackage{titlesec}
\titleformat{\section}
    [display]                   % форма
    {\filcenter\bfseries}       % формат полностью
    {ПРИЛОЖЕНИЕ \thesection}    % метка
    {}                          % отступ от метки
    {}                          % код перед телом

% раздел
\newcommand{\sect}[1]{%
    \clearpage
    \setcounter{figure}{0}  % сбросить нумерацию внутри раздела
    \setcounter{table}{0}
    \setcounter{listing}{0}
    \subsection{#1}
    \renewcommand{\theparagraph}{\thesubsection.\arabic{paragraph}}%
}
\titleformat{\subsection}{\filright\bfseries}{}{}{\thesubsection\hspace{1em}}
\titlespacing*{\subsection}
    {\parindent}    % отступ слева
    {1.5em}              % сверху
    {1.5em}              % снизу
\renewcommand{\thesubsection}{\arabic{subsection}}

% подраздел
\usepackage{placeins}
\newcommand{\subsect}[1]{%
    \FloatBarrier
    \subsubsection{#1}
    \renewcommand{\theparagraph}{\thesubsubsection.\arabic{paragraph}}
}
\titleformat{\subsubsection}{\filright\bfseries}{}{}{\thesubsubsection\hspace{1em}}
\titlespacing*{\subsubsection}
    {\parindent}
    {1.5em}
    {1.5em}

% пункт
\newcommand{\parag}{
    \paragraph{}
}
\titleformat{\paragraph}[runin]{}{\theparagraph}{1em}{ }  % для отступа
\titlespacing*{\paragraph}{\parindent}{}{}

% подпункт
\newcommand{\subparag}{
    \subparagraph{}
}
\titleformat{\subparagraph}[runin]{}{\thesubparagraph}{1em}{ }
\titlespacing*{\subparagraph}{\parindent}{}{}

% содержание
\usepackage{etoc}
\setcounter{tocdepth}{3}

% глубина нумерации разделов
\setcounter{secnumdepth}{5}

% перечисления
\usepackage{enumitem}
\setlist{
    topsep=0,                   % отступ сверху и снизу списка
    partopsep=0,                % то же самое
    leftmargin=0,               % отступ слева
    labelsep=0,                 % отступ метки
    align=left,                 % выравнивание метки
    listparindent=\parindent,   % отступ первой строки абзаца
    itemsep=0,                  % отступ между элементами
    parsep=0                    % отступ между абзацами и элементами
}
\setlist[itemize]{
    label=--~,  % в списках тире короткое, в тексте - длинное
    labelwidth=1.2em,
    itemindent=\parindent+\labelwidth
}
\setlist[enumerate]{
    label=\arabic*),
    labelwidth=1.4em,
    itemindent=\parindent+\labelwidth
}

% перечисление с буквенными метками
\AddEnumerateCounter*{\asbuk}{\c@asbuk}{7}
\newlist{asblist}{enumerate}{2}
\setlist[asblist]{
    label=\asbuk*),
    labelwidth=1.4em,
    itemindent=\parindent+\labelwidth
}

% подписи
\usepackage[singlelinecheck=false]{caption}
\DeclareCaptionLabelSeparator{gost}{~---~}
\captionsetup{labelsep=gost}

% иллюстрация
\newcommand{\fig}[3][1]{
    \begin{figure}[H]
        \centering
        \includegraphics[width=#1\textwidth]{#2}
        \caption{#3}\label{#2}
    \end{figure}
}
\renewcommand{\thefigure}{\thesubsection.\arabic{figure}}
\DeclareCaptionLabelFormat{gostfigure}{Рисунок #2}
\captionsetup[figure]{justification=centering, labelformat=gostfigure, position=bottom}
% font=singlespacing по умолчанию
%skip=-6pt

% листинг
\usepackage[newfloat, cache=false]{minted}
\newcommand{\lst}[2]{
    \begin{listing}[H]
        \centering
        \caption{#2}\label{#1}
        \begin{minipage}[t]{.8\textwidth}
            \inputminted[
                fontsize=\small,
                frame=single,
                breaklines,
                linenos
            ]{text}{#1}
        \end{minipage}
    \end{listing}
}
\renewcommand{\thelisting}{\thesubsection.\arabic{listing}}
\DeclareCaptionLabelFormat{custlisting}{Листинг #2}
\captionsetup[listing]{justification=raggedright, labelformat=custlisting, position=top}

% размер номера строки
\renewcommand{\theFancyVerbLine}{\rmfamily{\small \oldstylenums{\arabic{FancyVerbLine}}}}

% код в документе
\newenvironment{codepiece}[2]
{
    \VerbatimEnvironment
    \begin{listing}[H]
        \centering
        \caption{#2}\label{lst:#1}
        \begin{minipage}[t]{.8\textwidth}
            \begin{minted}[
                fontsize=\small,
                frame=single,
                breaklines,
                linenos
            ]{text}
}{
            \end{minted}
        \end{minipage}
    \end{listing}
}

% таблица
\newenvironment{tbl}[3]
{
    \begin{table}[H]
        \small
        \centering
        \caption{#2}\label{tbl:#1}
        \begin{tabular}{|#3|}
            \hline
}{
            \hline
        \end{tabular}
    \end{table}
}
\renewcommand{\thetable}{\thesubsection.\arabic{table}}
\DeclareCaptionLabelFormat{gosttable}{Таблица #2}
\captionsetup[table]{justification=raggedright, labelformat=gosttable, position=top}

\usepackage{tabularx}

% объединение строк
\usepackage{multirow}
\newcommand{\mr}[2]{\multirow[t]{#1}{=}{#2}}

% колонки
\usepackage{array}
\newcolumntype{M}[1]{>{\centering\arraybackslash}m{#1}}
\newcolumntype{N}[1]{>{\raggedright\arraybackslash}p{#1}}

% заголовок таблицы
\usepackage{xparse}
\NewExpandableDocumentCommand\thead{t< t> O{1} m m}{%
    \IfBooleanTF{#1}{%
        \IfBooleanTF{#2}{%
            \multicolumn{#3}{|M{#4}|}{#5}%
        }{%
            \multicolumn{#3}{|M{#4}}{#5}%
        }
    }{%
        \IfBooleanTF{#2}{%
            \multicolumn{#3}{M{#4}|}{#5}%
        }{%
            \multicolumn{#3}{M{#4}}{#5}%
        }%
    }%
}

% код в таблице
\newenvironment{tabcode}[1]
{
    \VerbatimEnvironment
    \begin{minipage}[t]{#1\textwidth}
    \begin{minted}[fontsize=\small, breaklines]{text}
}{
    \end{minted}
    \end{minipage}
}

% длинная таблица
\usepackage{longtable}
\newenvironment{longtbl}[3]
{
    \small
    \begin{longtable}{|#3|}
        \caption{#2}\label{tbl:#1}\\
        \hline
}{
        \hline
    \end{longtable}
}

% математика
\usepackage{mathtools}  % amsmath
\numberwithin{equation}{subsection}

% графики
\usepackage{tikz, pgfplots}
\pgfplotsset{compat=newest}

\usepackage{csquotes}
\usepackage{adjustbox}
\usepackage{float}
\usepackage{url}

% источники
\usepackage[%
    backend=biber,%
    bibstyle=gost-numeric%
]{biblatex}
\addbibresource{bibliography.bib}
\newcommand{\showbib}{%
    \structel{СПИСОК ИСПОЛЬЗОВАННЫХ ИСТОЧНИКОВ}%
    \printbibliography[heading=none]%
}

% отступы в источниках
\defbibenvironment{bibliography}
    {\list
        {}
        {\setlength{\leftmargin}{0}%
         \setlength{\itemindent}{\parindent}%
         \setlength{\itemsep}{0}%
         \setlength{\parsep}{0}}}
    {\endlist}
    {\item
     \printtext[labelnumberwidth]{%
        \printfield{labelprefix}%
        \printfield{labelnumber}%
     }%
     \hspace{0.5em}}

% метка без точки
\DeclareFieldFormat{labelnumberwidth}{#1}

% номер последней страницы
\usepackage{lastpage}

% счетчик источников
\newtotcounter{bibcount}
\AtEveryBibitem{
    \stepcounter{bibcount}%
}

% счетчики таблиц и рисунков
\usepackage{xassoccnt}
\newtotcounter{tblcount}
\DeclareAssociatedCounters{table}{tblcount}
\newtotcounter{figcount}
\DeclareAssociatedCounters{figure}{figcount}

%% для отладки
%%\usepackage{showframe}
%%\renewcommand\ShowFrameLinethickness{0.25pt}
%%\renewcommand*\ShowFrameColor{\color{red}}
%%\usepackage{graphicx}

\begin{document}

\subsect{INTRO}
Уважаемый председатель государственной экзаменационной комиссии, уважаемые
члены государственной экзаменационной комиссии и уважаемые присутствующие!
Позвольте представить Вашему вниманию мою выпускную квалификационную работу

\subsect{Актуальность}
Актуальность моего приложения обусловлена множеством аспектов, но ключевые:
возрастающее количество пользователей языковых моделей для бизнесом
(мне было интересно собственноручно создать такой проект), в свою очередь
актуальность для ВУЗа --- создание универсального ассистента, способного
отвечать на заданые вопросы без хардкодинга и с возможностью неявного поиска.

\subsect{Цели и задачи}

Передо мной была поставлена следующая цель: спроектировать и разработать
приложение для взаимодействия с структурой предприятия посредством какого-либо
интерфейса. Был выбран интерфейс в виде телеграмм бота, но мою систему можно
без особых сложностей подключить как к ВКонтакте или Яндекс.Алисе. В настоящий
момент работает только в Telegram.

Поставленные задачи отображены на слайде.

\subsect{Сравнительный анализ}

Во время начала разработки моей системы непосредственных конкурентов у 
моей системы не было, поэтому я сравнивал свою систему с похожими решениями.
Недавно начало разрабатываться подобное решение от компании Сбербанк.

\subsect{Swot}

Для выявления сильных строн и потенциальных угроз был проведен SWOT анализ.

\subsect{Временные оценки}

На данном сладе расположена оценка временных затрат, а так же затрат на
разработку. 

\subsect{Гант}

На данном сладе расположена диаграмма ганта с визуализацией этапов разработки

\subsect{Функциональное моделирование}

На данном слайде схематично отображены функциональные требования.
Таким образом система должна уметь: приветствовать пользователя, уметь
объяснить пользователю как управлять системой, а так же предоставлять
информацию о учреждении.

\subsect{Схема работы приложения}

Пользователь отправляет запрос сервису, после чего запрос преобразуется в
вектор. Для преобразования вектора испольуется малослойная нейронная сеть.
Далее полученный вектор сравнивается при помощи вычисления косинусного и 
Евклидового растояний. Вектора с наибольшим коэфициентом подобия достаются
из базы знаний и подставляются в промт. Далее промт с данными из базы знаний
и запросом от пользователя отправляется на вход языковой модели, после чего
происходит генерация ответа, основываясь на предоставленном контексте.
После окончания генерации ответа происходит отправка ответа пользователю.

\subsect{Используемые технологии}

В основном использовалось три технологии: Python для разработки сервиса и RAG,
PyTorch для глубокого обучения, в виду того, что многие методы удобно 
написаны на языке C++, который во много раз быстрее Python, а так же Docker
для упаковки приложения в контейнер для сборки всех необходимых зависмостей,
что позволяет запустить приложение на любом устройстве с любой операционной
системой.

\subsect{Векторизация}

Векторизация происходит при помощи малослойной нейронной сети. На вход подается
текстовый корпус. Для оптимизации используется градиентный спуск.

\subsect{Сопоставление векторов}

Сопоставление вектор происходит при помощи расчета Евклидового расстояния между
векторами. На слайде можно наблюдать визуализацию данного метода для K=5. 
В моей системе используется K=3, чтобы уместиться в контекстное окно 
языковой модели.

\subsect{Расчет экономической эффективности}

Для расчета экономической эффективности использовалась предоставленная на
слайде формула. Процент окупаемости составил 300 процентов, что я считаю
довольно хорошим результатом.

\subsect{Реализация ИС}

На слайдах можно наблюдать реализацию системы. Работа приложения отвечает
описанным функциональным требованиям.
Так же планировался блок для сурдо-перевода, но у меня не было такого
количества данных для тренировки модели + это очень трудозатратно.
Посмотреть работы приложения можно отсканировав QR-код. Приложение работает
на моем персональном компьютере и возможно у меня кончится видеопамять)

\end{document}
